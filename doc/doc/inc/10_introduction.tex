% -*- coding: UTF-8 -*-
% vim: autoindent expandtab tabstop=4 sw=4 sts=4 filetype=tex
% vim: spelllang=de
% chktex-file 27 - disable warning about missing include files

\chapter{Einleitung}
\label{chap:10_introduction}

In~\cite{osterwalder_sven_volume_2016} wurde mit Sphere-Tracing ein hoch
effizientes Ray-Tracing-Verfahren vorgestellt, welches Objekte (und somit auch
Szenen) sowie Operationen mittels impliziten Funktionen darstellt. Dies mag auf
den ersten Blick praktisch erscheinen, da die Definition der meisten Objekte
und Operationen nur wenige Zeilen lang ist. Dies hat jedoch zur Folge, dass
einerseits Programme selbst bei kleinsten Änderungen neu kompiliert werden
müssen und, dass andererseits bei komplexen Szenen schnell die Übersicht
verloren geht. Möchte man Szenen beziehungsweise Objekte animieren, so erhöht
dies die Komplexität erneut.

Diese Projektarbeit stellt daher eine Software-Architektur für ein System zur
einfachen Erstellung und Handhabung visueller Szenen in Echtzeit vor.

Das System erlaubt die Erstellung und Handhabung von Szenen mittels einer
grafischen Benutzeroberfläche (GUI). Ein Graph erlaubt eine einfache und
intuitive Komposition von visuellen Szenen. Ein Sequenzer erlaubt die Animation
von Elementen, wie z.B. Modellen oder Bitmaps.

Als Renderingverfahren kommt Sphere-Tracing
zum Einsatz, ein hoch-optimiertes Ray-Tracing-Verfahren, welches die
Darstellung von Szenen in Echtzeit per GPU erlaubt.

Die Machbarkeit des Konzeptes wird durch einen Prototypen aufgezeigt, welcher
die Komposition sowie Darstellung von einfachen Szenen in Echtzeit erlaubt.
