% -*- coding: UTF-8 -*-
% vim: autoindent expandtab tabstop=4 sw=4 sts=4 filetype=tex
% vim: spelllang=de spell
% chktex-file 27 - disable warning about missing include files

\section{Komponenten}
\label{sec:prototype:components}

Durch die vorhergehende Projektarbeit --- MTE7101
(siehe~\cite{osterwalder_sven_volume_2016}) --- sowie persönlicher Erfahrungen
bei diversen Projekten, wählte der Autor zur Umsetzung des Prototypen die
Software in der
Tabelle~\ref{table:requirements:additional-requirements:software} zur
Umsetzung. Vor dieser Projektarbeit führte der Autor Recherchen betreffend
Bibliotheken zur Erstellung und Darstellung von grafischen Benutzeroberflächen
durch. Zuerst wurde Qt~\footnote{https://www.qt.io} in Betracht gezogen. Diese
Bibliothek ist sehr umfangreich und bietet bereits viele der benötigten
Merkmale. Allerdings hat sie den Nachteil, dass sie relativ gross ist und den
Prototypen aufbläht.

Der Autor stiess schliesslich auf
NanoGUI~\footnote{https://github.com/wjakob/nanogui} von Jakob Wenzel. Es
handelt sich dabei um eine minimalistische, plattformübergreifende (Grafik-)
Bibliothek für OpenGL 3.x~\cite{jakob_wenzel_wjakob/nanogui:_2016}.  Die
Bibliothek basiert auf NanoVG~\footnote{https://github.com/memononen/NanoVG}
von Mikko Mononen. NanoGUI bietet automatische Layout-Erzeugung, C++11
Lambda-Callbacks, diverse Widget-Typen und Retina-fähiges
Rendering~\cite{jakob_wenzel_wjakob/nanogui:_2016}. Da dies genau den
Anforderungen entsprechen zu schien, entschloss sich der Autor für die
Verwendung von NanoGUI.\@

Während der Arbeit am Prototypen stellte sich jedoch heraus, dass viele der
benötigten Komponenten nicht vorhanden sind und so mussten diese erst selbst
entwickelt werden, so etwa der gesamte Graph inklusive Knoten, Ein- und
Ausgängen und Verbindungen zwischen Knoten. Auch ein Kontext-Menü und einzelne,
anwählbare Links (Text, welcher nach Klick eine Aktion ausführt) waren nicht
vorhanden und mussten entwickelt werden.

Dies führte dazu, dass viel Zeit in die Entwicklung des Prototypen gesteckt
wurde. So lag der Fokus anfangs primär darin einen lauffähigen Prototypen zu
erhalten, anstatt zuerst ein (Software-) Design zu erstellen. Dies hatte
schliesslich zur Folge, dass keine saubere Trennung (zum Beispiel in Schichten)
stattfand und der Code weniger erweiterbar und wartbar wurde.

Die so gesammelten Erfahrungen flossen schliesslich in die Architektur für das
darauffolgende Projekt (\ref{chap:software-architecture}) ein. So fiel
letztendlich der Entscheid doch wieder auf die Verwendung der Qt-Bibliothek zu
Ungunsten von NanoGUI.\@ Qt bietet eine viel umfangreichere Basis und beinhaltet
viele der benötigten Komponenten und Features bereits.

% Sigh..
% FYI:  Tabular and a fixed text-width of 79 sucks!
% FYI2: set cursorcolumn with a lot of text sucks too!

\begin{longtabu}{p{0.2\textwidth}p{0.1\textwidth}p{0.6\textwidth}p{0.1\textwidth}}
    \caption{Verwendete
        Software/Technologien}\label{table:prototype:software}\\
    \toprule
    \textbf{Komponente} & \textbf{Version} & \textbf{Beschreibung} & \textbf{Verweise} \\
    \midrule
    C++        & 11      & Objektorientierte Programmiersprache
                           & \protect\footnotemark\footnotetext{
                               \url{http://www.iso.org/iso/iso_catalogue/catalogue_tc/catalogue_detail.htm?csnumber=50372}
                           }\\

    OpenGL     & 4.5     & Plattformunabhängige Programmierschnittstelle zur
                           Entwicklung von 2D- und 3D-Computergrafikanwendungen
                           \parencite{wikipedia_the_free_encyclopedia_opengl_2015}
                           &\protect\footnotemark\footnotetext{
                               \url{https://www.opengl.org/registry/doc/glspec45.core.pdf}
                           }\\

    NanoGUI    & ---     & ``NanoGUI is a a minimalistic cross-platform widget
                             library for OpenGL 3.x. It supports automatic
                             layout generation, stateful C++11 lambdas
                             callbacks, a variety of useful widget types and
                             Retina-capable rendering on Apple devices thanks
                             to NanoVG by Mikko Mononen. Python bindings of all
                             functionality are provided using
                             pybind11.''~\parencite{jakob_wenzel_wjakob/nanogui:_2016}
                           &\protect\footnotemark\footnotetext{
                               \url{https://github.com/wjakob/nanogui}
                           }\\

    GLFW       & 3.1.2   & OpenGL-Bibliothek, welche die Erstellung und
                           Verwaltung von Fenstern sowie OpenGL-Kontexte
                           vereinfacht
                           \parencite{wikipedia_the_free_encyclopedia_glfw_2015}.
                           &\protect\footnotemark\footnotetext{
                               \url{http://www.glfw.org}
                           }\\

    GLEW       & 1.13    & OpenGL Extension Wrangler. Bibliothek zum Abfragen
                           und Laden von OpenGL-Erweiterungen (Extensions)
                           \parencite{wikipedia_the_free_encyclopedia_opengl_2015-1}.  % chktex 8 -- wrong dash-length
                           &\protect\footnotemark\footnotetext{
                               \url{http://glew.sourceforge.net}
                           }\\

    Eigen      & 3.2.8   & ``Eigen is a C++ template library for linear
                            algebra: matrices, vectors, numerical solvers, and
                            related algorithms''\parencite{benoit_jacob_eigen_2016}
                           &\protect\footnotemark\footnotetext{
                               \url{http://eigen.tuxfamily.org}
                           }\\

    CMake      & 3.3.2   & Software zur Verwaltung von Build-Prozessen von 
                           Software
                           &\protect\footnotemark\footnotetext{
                               \url{https://www.cmake.org}
                           }\\

    LLVM       & 3.8.1   & Ansammlung von modularen und wiederverwendbaren
                           Compilern und Toolchains.
                           &\protect\footnotemark\footnotetext{
                               \url{http://llvm.org}
                           }\\

    Clang      & 3.8.1   & Compiler-Frontend für LLVM.\@
                           &\protect\footnotemark\footnotetext{
                               \url{http://clang.llvm.org}
                           }\\

    Boost      & 1.59.0  & Freie Bibliothek bestehend aus einer Vielzahl von
                           Bibliotheken, die den unterschiedlichsten
                           Aufgaben von Algorithmen auf Graphen über 
                           Metaprogrammierung bis hin zu Speicherverwaltung
                           dienen
                           \parencite{wikipedia_the_free_encyclopedia_boost_2015}.
                           &\protect\footnotemark\footnotetext{
                               \url{http://www.boost.org}
                           }\\
    \bottomrule
\end{longtabu}

