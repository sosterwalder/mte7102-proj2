% -*- coding: UTF-8 -*-
% vim: autoindent expandtab tabstop=4 sw=4 sts=4 filetype=tex
% vim: spelllang=de spell
% chktex-file 27 - disable warning about missing include files

\subsection{Beleuchtungsmodell}
\label{sec:rendering_implicit_surfaces_lighting}

Um implizite Oberflächen darstellen zu können, muss ein
Beleuchtungsmodell gewählt werden. Andernfalls wäre  das dargestellte
Bild nur schwarz. Zur Vereinfachung wird im Rahmen dieser Projektarbeit
das in~\autoref{subsec:local_illumination_models} vorgestellte
Phong-Beleuchtungsmodell verwendet.

Daher wird die resultierende Farbe eines Punktes im Raum $I(\bm{x})$ aus
ambienten, diffusen und reflektierenden Anteilen berechnet:

\begin{gather}
    I(\bm{x}) = I_{\text{ambient}} + I_{\text{diffuse}} + I_{\text{specular}}
\end{gather}

Wie bereits zuvor in~\autoref{subsec:local_illumination_models} erwähnt,
wird der emissive Term bewusst weggelassen, da keine emissiven Materialen
dargestellt werden sollen. Als Lichtquelle wird eine einzelne direktionale
Lichtquelle gewählt. Analog zu den vorherigen Abschnitten ist $\bm{x}$
in den folgenden Abschnitten ein Punkt $(x, y, z)$ auf einer impliziten
Oberfläche $A$.

Der \textit{ambiente Anteil} $I_{\text{ambient}}$ ergibt sich dann wie
folgt:

\begin{gather}
    \text{I}_{\text{ambient}} = k_{\text{ambient}}(\bm{x}) \cdot L_{\text{ambient}}
\end{gather}

Dabei ist $k_{\text{ambient}}(\bm{x})$ der ambienten Faktor des
Punktes $\bm{x}$ und $L_{\text{ambient}}$ die Farbe des eingehenden ambienten
Lichtes~\parencites[S. 723]{glassner_introduction_1989}[Kapitel 5, Abschnitt 5.2.1]{fernando_cg_2003}.

Der \textit{diffuse Anteil} $I_{\text{diffuse}}$ ergibt sich wie folgt:

\begin{gather}
    I_{\text{diffuse}} = k_{\text{diffuse}}(\bm{x}) \cdot L_{\text{diffuse}} \cdot \max(\bm{n} \cdot \bm{l}, 0)
\end{gather}

Dabei ist $k_{\text{diffuse}}(\bm{x})$ der diffusen Faktor am Punkt $\bm{x}$
und $L_{\text{diffuse}}$ die Farbe des eingehenden diffusen Lichtes.
Die Richtung der Lichtquelle, ausgehend von Punkt $\bm{x}$, ergibt sich
durch das Punktprodukt zwischen der Einheitsnormalen $\bm{n}$ der
Oberfläche an dem Punkt und der Richtung der Lichtquelle
$\bm{l}$~\parencites[S. 724]{glassner_introduction_1989}[Kapitel 5, Abschnitt 5.2.1]{fernando_cg_2003}.

Der \textit{reflektierende Anteil} $I_{\text{specular}}$ ergibt sich
wie folgt:

\begin{gather}
    I_{\text{specular}} = n_{\text{facing}} \cdot k_{\text{specular}}(\bm{x}) \cdot L_{\text{specular}} \cdot \max{(\bm{n} \cdot \bm{h}, 0)}^{k_{e}}
\end{gather}

Dabei ist $k_{\text{specular}}(\bm{x})$ der reflektierenden Faktor des
Punktes $\bm{x}$ und $L_{\text{specular}}$ die Farbe des eingehenden
Lichtes. Bei $\bm{h}$ handelt es sich um einen Einheitsvektor,
welcher in der Hälfte zwischen der Blickrichtung des Betrachters bzw.\
der Kamera ($\vv{V}$) und $\bm{l}$ der Richtung der Lichtquelle
ausgehend von dem Punkt $\bm{x}$ ist~\parencite[S.
731]{glassner_introduction_1989}. Der Exponent $k_{e}$ gibt an, wie rau
bzw.\ wie spiegelnd die Oberfläche am Punkt $\bm{x}$
ist~\parencite[Kapitel 5, Abschnitt 5.2.1]{fernando_cg_2003}. Der Faktor
$n_{\text{facing}}$ definiert, ob die Oberfläche überhaupt einen
reflektierenden Anteil hat~\parencite[Kapitel 5, Abschnitt 5.2.1]{fernando_cg_2003}:

\begin{equation}
    n_{\text{facing}} = \begin{cases}
        0 & \quad \text{if } \bm{n} \cdot \bm{l} \leq 0\\
        1 & \quad \text{if } \bm{n} \cdot \bm{l} > 0 \\
    \end{cases}
\end{equation}

Für die Berechnung der Intensität des Lichtes bzw.\ der Farbe einer Oberfläche
wird die Normale der Oberfläche benötigt. Gemäss~\citeauthor{hart_ray_1989}
kann diese mittels des Gradienten des Distanzfeldes
eines Punktes einer impliziten Oberfläche berechnet
werden~\parencite[S. 292 bis 293]{hart_ray_1989}:

\begin{gather}
    \bm{n}_{x} = f(x + \varepsilon, y, z) - f(x - \varepsilon, y, z) \\
    \bm{n}_{y} = f(x, y + \varepsilon,  z) - f(x, y - \varepsilon,  z) \\
    \bm{n}_{z} = f(x, y, z + \varepsilon) - f(x, y, z - \varepsilon) \\
\end{gather}

Dabei ist $\bm{n} = \begin{bmatrix} x_{n} \\ y_{n} \\ z_{n} \end{bmatrix}$
die Normale der Oberfläche in Form eines Vektors und $f$ eine
Distanzfunktion~\parencites[S. 292 bis 293]{hart_ray_1989}[S.
13]{hart_ray_1993}.

Der Gradient einer Funktion $f: \mathbb{R}^{n} \to \mathbb{R}$ wird wie
folgt berechnet:

\begin{gather}
    \text{grad}(f) = \nabla f = \begin{bmatrix}
        \frac{\partial f}{\partial x_{1}} \\
        \frac{\partial f}{\partial x_{2}} \\
        \vdots \\
        \frac{\partial f}{\partial x_{n}} 
    \end{bmatrix}\\
    \text{grad}(f) = f_{x}\bm{i} + f_{y}\bm{j} + f_{z}\bm{k}\\
\end{gather}

Dabei sind $\bm{i}$, $\bm{j}$ und $\bm{k}$ Vektoren der Form $\begin{bmatrix}
    x \\ y \\ z \end{bmatrix}$.

\citeauthor{hart_ray_1989} geben $\varepsilon$ als die minimale
Inkrementation eines (Licht-) Strahles an und definieren diese als
Funktion zur Bestimmung der Sichtbarkeit $\Gamma_{\alpha,
    \delta}$~\parencite[S. 293]{hart_ray_1989}:

\begin{gather}
    \Gamma(d) = \alpha d^{\delta}
\end{gather}

in Abhängigkeit der euklidischen Distanz $d$ des Betrachters respektive
der Kamera zur aktuellen Position des (Licht-) Strahles:

\begin{gather}
    d = |r_{\bm{n}} - r_{0}|
\end{gather}

Dabei ist $\delta$ ein so genannter ``depth-cueing''-Exponent
(``depth-cueing'' oder auch ``foldback'': ``a process for returning a
signal to a performer
instantly''~\parencite{liam_collins_sons_&_co._ltd._collins_2015}) und
$\alpha$ ein empirischer Anteil, welcher die Tiefenauflösung des
Objektes definiert.  Details dazu finden sich
unter~\cite[S. 293, Abschnitt 4.2 --- ``Clarity'']{hart_ray_1989}.

Es folgt also:

\begin{gather}
    \varepsilon = \Gamma_{\alpha, \delta}(|r_{\bm{n}} - r_0|)
\end{gather}

Die Korrektheit der Berechnung der Normalen $\bm{n}$ hängt von der
Grösse von $\varepsilon$ ab. Daher wird für gewöhnlich ein kleiner Wert
für $\varepsilon$ gewählt~\parencite[S. 293]{hart_ray_1989}.

Die Normale der Oberfläche sollte schliesslich noch normalisiert werden.

Liefert die oben genannte Gradiente, bestehend aus 6 Punkten, eine zu
geringe Genauigkeit, so kann diese gemäss \citeauthor{hart_ray_1989}
erweitert werden~\parencite[S. 293]{hart_ray_1989}.\\
Die Erweiterung erfolgt durch Hinzunahme von Punkten, welche eine
gemeinsame Kante haben. Dies erzeugt eine Gradiente bestehend aus 18
Punkten. Werden noch die Punkte hinzugenommen, welche gemeinsame
Eckpunkte haben, so ergibt sich eine Gradiente bestehend aus 26
Punkten~\parencite[S. 293]{hart_ray_1989}.
