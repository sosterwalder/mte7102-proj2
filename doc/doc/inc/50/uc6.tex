% -*- coding: UTF-8 -*-
% vim: autoindent expandtab tabstop=4 sw=4 sts=4 filetype=tex
% vim: spelllang=de spell
% chktex-file 27 - disable warning about missing include files

\subsubsection{Use Case UC6: Erstellen einer neuen Szene}
\label{ssubsec:requirements:use-cases:uc6}

\begin{longtabu}{p{0.5\textwidth}p{0.5\textwidth}}
    \centering\\
    \caption{Use Case UC6: Erstellen einer neuen
        Szene.}\label{table:uc6-create-new-scene}\\
    \toprule
        \textbf{Bereich} &
        Editor \\
    \cmidrule(r){1-1}\cmidrule(lr){2-2}
        \textbf{Stufe (level)} &
        Ziel des Benutzers \\
    \cmidrule(r){1-1}\cmidrule(lr){2-2}
        \textbf{Primärer Aktor} &
        Anwender \\
    \cmidrule(r){1-1}\cmidrule(lr){2-2}
        \textbf{Stakeholder und Interessen} &
        Anwender: Möchte eine neue Szene einer Echtzeit-Animation erstellen.\newline
        Betrachter: Möchte eine erstellte Szene ansehen.\\
    \cmidrule(r){1-1}\cmidrule(lr){2-2}
        \textbf{Vorbedingungen} &
        \begin{itemize}
            \item{Die Editor-Applikation ist gestartet.}
        \end{itemize} \\
    \cmidrule(r){1-1}\cmidrule(lr){2-2}
        \textbf{Erfolgsszenario} &
        \begin{enumerate}
            \item{Der Anwender startet die Editor-Applikation.}
            \item{Der Anwender klickt mit der rechten Maustaste die Root-Szene in der
                    Bibliothek an.}
            \item{Der Anwender wählt im Kontext-Menü den Menüpunkt zum Erstellen einer neue Szene.}
            \item{Der Editor erstellt mittels dem Szenegraphen eine neue
                    Szene und fügt diese der Liste von Szenen hinzu.}
            \item{Die Szene wird entsprechend im Szenegraphen als
                    Unterknoten des Root-Knotens dargestellt.}
        \end{enumerate} \\
    \cmidrule(r){1-1}\cmidrule(lr){2-2}
        \textbf{Erweiterungen} &
        \begin{enumerate}[label= (\alph*)]
            \item{Schliessen des Editors bei gemachten Änderungen ohne zu
                    speichern
                \begin{enumerate}[label= (\roman*)]
                    \item{Der Anwender schliesst den Editor bei gemachten
                            Änderungen ohne diese zu speichern.}
                    \item{Der Editor weist den Anwender auf die
                            nicht gespeicherten Änderungen hin und bietet
                            die Möglichkeit diese zu speichern, diese nicht
                            zu speichern oder das Schliessen abzubrechen.}
                \end{enumerate}
            }
        \end{enumerate} \\
    \cmidrule(r){1-1}\cmidrule(lr){2-2}
    \textbf{Fehlerfall} &
        \begin{enumerate} \\
            \item{Absturz/Ausfall des Systems
                \begin{enumerate}[label= (\roman*)]
                        \item{Das System stürzt an einem beliebigen Punkt
                                ab.}
                        \item{Neustart des Editors.}
                        \item{Der Editor stellt den ihm zuletzt bekannten
                                Punkt der Animation wieder her.}
                \end{enumerate}
            }
            \item{Die Root-Szene ist nicht in der Bibliothek vorhanden
                \begin{enumerate}[label= (\roman*)]
                    \item{Die Root-Szene ist nicht in der Bibliothek vorhanden.}
                    \item{Der Anwender wird via Dialog-Fenster über den Fehler
                            informiert.}
                    \item{Der Editor wird geschlossen.}
                \end{enumerate}
            }
            \item{Es erscheint kein Kontext-Menü
                \begin{enumerate}[label= (\roman*)]
                    \item{Der Anwender wird via Dialog-Fenster über den Fehler
                            informiert.}
                    \item{Der Editor wird geschlossen.}
                \end{enumerate}
            }
            \item{Es kann keine neue Szene erstellt werden
                \begin{enumerate}[label= (\roman*)]
                    \item{Beim Erstellen der neuen Szene tritt ein Fehler auf.}
                    \item{Der Anwender wird via Dialog-Fenster über den Fehler
                            informiert.}
                \end{enumerate}
            }
        \end{enumerate} \\
    \cmidrule(r){1-1}\cmidrule(lr){2-2}
        \textbf{Zusätzliche Anforderungen} &
        --- \\
    \bottomrule
\end{longtabu}
