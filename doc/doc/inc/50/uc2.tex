% -*- coding: UTF-8 -*-
% vim: autoindent expandtab tabstop=4 sw=4 sts=4 filetype=tex
% vim: spelllang=de spell
% chktex-file 27 - disable warning about missing include files

\subsubsection{Use Case UC2: Erstellen einer Echtzeit-Animation}
\label{ssubsec:requirements:use-cases:uc2}

\begin{longtabu}{p{0.5\textwidth}p{0.5\textwidth}}
    \centering\\
    \caption{Use Case UC2: Erstellen einer
        Echtzeit-Animation.}\label{table:uc2-create-demo}\\
    \toprule
        \textbf{Bereich} &
        Editor \\
    \cmidrule(r){1-1}\cmidrule(lr){2-2}
        \textbf{Stufe (level)} &
        Ziel des Benutzers \\
    \cmidrule(r){1-1}\cmidrule(lr){2-2}
        \textbf{Primärer Aktor} &
        Anwender \\
    \cmidrule(r){1-1}\cmidrule(lr){2-2}
        \textbf{Stakeholder und Interessen} &
        Anwender: Möchte eine neue Echtzeit-Animation erstellen.\newline
        Betrachter: Möchte eine erstellte Echtzeit-Animation ansehen. \\
    \cmidrule(r){1-1}\cmidrule(lr){2-2}
        \textbf{Vorbedingungen} &
        Der Editor sichert eine Animation alle \textit{n}-Minuten in einer
        temporären Sicherungsdatei. \\
    \cmidrule(r){1-1}\cmidrule(lr){2-2}
        \textbf{Erfolgsszenario} &
        \begin{enumerate}
            \item{Der Anwender startet die Editor-Applikation.}
            \item{Der Anwender fügt Elemente zu einer Szene zusammen.}
            \item{Der Anwender legt Start und Ende einer Animation fest.}
            \item{Der Anwender speichert die getätigte Arbeit.}
            \item{Der Anwender exportiert die erstellte Animation.}
            \item{Der Anwender schliesst den Editor.}
        \end{enumerate} \\
    \cmidrule(r){1-1}\cmidrule(lr){2-2}
        \textbf{Erweiterungen} &
        \begin{enumerate}[label= (\alph*)]
            \item{Schliessen des Editors bei gemachten Änderungen ohne zu
                    speichern
                \begin{enumerate}[label= (\roman*)]
                    \item{Der Anwender schliesst den Editor bei gemachten
                            Änderungen ohne diese zu speichern.}
                    \item{Der Editor weist den Anwender auf die
                            nicht gespeicherten Änderungen hin und bietet
                            die Möglichkeit diese zu speichern, diese nicht
                            zu speichern oder das Schliessen abzubrechen.}
                \end{enumerate}
            }
            \item{Absturz/Ausfall des Systems
                \begin{enumerate}[label= (\roman*)]
                        \item{Das System stürzt an einem beliebigen Punkt
                                ab.}
                        \item{Neustart des Editors.}
                        \item{Der Editor stellt den ihm zuletzt bekannten
                                Punkt der Animation wieder her.}
                \end{enumerate}
            }
        \end{enumerate}
        \\
    \cmidrule(r){1-1}\cmidrule(lr){2-2}
        \textbf{Zusätzliche Anforderungen} &
        Betreffend dem Exportieren ist das zu verwendende Format zum jetzigen
        Zeitpunkt noch unklar. In Frage käme zum Beispiel
        JSON\protect\footnotemark{} oder X3D\protect\footnotemark{}.\\
        \footnotetext{http://www.json.org/}
        \footnotetext{http://www.web3d.org/x3d}
    \bottomrule
\end{longtabu}
