% -*- coding: UTF-8 -*-
% vim: autoindent expandtab tabstop=4 sw=4 sts=4 filetype=tex
% vim: spelllang=de spell
% chktex-file 27 - disable warning about missing include files

\subsubsection{Use Case UC8: Verbinden eines Knotens im Graphen einer Szene}
\label{ssubsec:requirements:use-cases:uc8}

\begin{longtabu}{p{0.5\textwidth}p{0.5\textwidth}}
    \centering\\
    \caption{Use Case UC8: Verbinden eines Knotens im Graphen einer
        Szene.}\label{table:uc8-connect-node}\\
    \toprule
        \textbf{Bereich} &
        Editor \\
    \cmidrule(r){1-1}\cmidrule(lr){2-2}
        \textbf{Stufe (level)} &
        Ziel des Benutzers \\
    \cmidrule(r){1-1}\cmidrule(lr){2-2}
        \textbf{Primärer Aktor} &
        Anwender \\
    \cmidrule(r){1-1}\cmidrule(lr){2-2}
        \textbf{Stakeholder und Interessen} &
        Anwender: Möchte (Teil-) Inhalt einer Szene einer
        Echtzeit-Animation sichtbar machen.\newline
        Betrachter: Möchte neu erstellten Inhalt ansehen.\newline
        Entwickler: Möchte, dass seine erstellten Knoten-Typen produktiv
        eingesetzt werden können. \\
    \cmidrule(r){1-1}\cmidrule(lr){2-2}
        \textbf{Vorbedingungen} &
        \begin{itemize}
            \item{Die Editor-Applikation ist gestartet.}
            \item{Es sind Knoten-Typen zum Hinzufügen vorhanden.}
            \item{Eine beliebige Szene ist angewählt.}
            \item{Die Szene verfügt bereits über Knoten.}
        \end{itemize} \\
    \cmidrule(r){1-1}\cmidrule(lr){2-2}
        \textbf{Erfolgsszenario} &
        \begin{enumerate}
            \item{Der Anwender klickt mit der linken Maustaste auf die
                    Ausgangs-Buchse des gewünschten Knotens und hält die
                    linke Maustaste gedrückt.}
                \item{Der Editor erstellt eine Verbindungslinie mit
                        Ursprung in der Ausgangs-Buchse des Knotens und der
                        aktuellen Position des Mauszeigers als Ziel.}
            \item{Der Anwender verschiebt den Mauszeiger über die
                    Eingangs-Buchse eines kompatiblen Knotens (zum Beispiel
                    einen Szene-Knoten zum Hauptausgangs-Knoten) und lässt
                    die linke Maustaste los.}
            \item{Der Editor erstellt eine neue Verbindung zwischen dem
                    dem Ausgangs- und dem Zielknoten.}
            \item{Der Editor fügt die neu erstellte Verbindung zur Ausgabe des
                    Szenegraphen bzw.\ der Szene hinzu.}
            \item{Der Editor sendet via Szenegraph das Signal, dass eine
                    neue Verbindung hinzugefügt wurde.}
            \item{Der Editor merkt sich, dass sein Fenster verändert
                    wurde.}
        \end{enumerate} \\
    \cmidrule(r){1-1}\cmidrule(lr){2-2}
        \textbf{Erweiterungen} &
        \begin{enumerate}[label= (\alph*)]
            \item{Schliessen des Editors bei gemachten Änderungen ohne zu
                    speichern
                \begin{enumerate}[label= (\roman*)]
                    \item{Der Anwender schliesst den Editor bei gemachten
                            Änderungen ohne diese zu speichern.}
                    \item{Der Editor weist den Anwender auf die
                            nicht gespeicherten Änderungen hin und bietet
                            die Möglichkeit diese zu speichern, diese nicht
                            zu speichern oder das Schliessen abzubrechen.}
                \end{enumerate}
            }
            \item{Absturz/Ausfall des Systems
                \begin{enumerate}[label= (\roman*)]
                        \item{Das System stürzt an einem beliebigen Punkt
                                ab.}
                        \item{Neustart des Editors.}
                        \item{Der Editor stellt den ihm zuletzt bekannten
                                Punkt der Animation wieder her.}
                \end{enumerate}
            } \end{enumerate}
        \\
    \cmidrule(r){1-1}\cmidrule(lr){2-2}
        \textbf{Zusätzliche Anforderungen} &
        --- \\
    \bottomrule
\end{longtabu}
