% -*- coding: UTF-8 -*-
% vim: autoindent expandtab tabstop=4 sw=4 sts=4 filetype=tex
% vim: spelllang=de spell
% chktex-file 27 - disable warning about missing include files

\subsection{Use Cases}
\label{subsec::requirements:use-cases}

Bei Use Cases handelt es sich um Anforderungen, genauer gesagt um funktionelle
bzw. Anforderungen an das Verhalten eines Systems~\cite{larman_applying_2004}.
Sie sagen also aus, was ein System tut bzw.\ tun soll. Die nachfolgenden Use
Cases sind keinesfalls vollständig --- dann wäre der Sinn des UP bzw.\ eines
iterativen Vorgehens verfehlt. Sie entwickeln sich viel mehr über die Zeit mit
dem Fortschreiten der Umsetzung.

% -*- coding: UTF-8 -*-
% vim: autoindent expandtab tabstop=4 sw=4 sts=4 filetype=tex
% vim: spelllang=de spell
% chktex-file 27 - disable warning about missing include files

\begin{table}[H]
    \centering
    \caption{Use Case UC1: Betrachten einer Echtzeit-Animation.}\label{table:uc1-watch-demo}
    \begin{tabular}{p{0.5\textwidth}p{0.5\textwidth}}
        \toprule
            \textbf{Bereich} &
            Player\\
        \cmidrule(r){1-1}\cmidrule(lr){2-2}
            \textbf{Level} &
            Ziel des Benutzers \\
        \cmidrule(r){1-1}\cmidrule(lr){2-2}
            \textbf{Primärer Aktor} &
            Betrachter \\
        \cmidrule(r){1-1}\cmidrule(lr){2-2}
            \textbf{Stakeholder und Interessen} &
            Betrachter: Möchte eine zuvor erstellte Echtzeit-Animation ansehen.  \newline
            Anwender: Testen einer erstellen Echtzeit-Animation. \newline
            Entwickler: Testen einer erstellen Echtzeit-Animation. \\
        \cmidrule(r){1-1}\cmidrule(lr){2-2}
            \textbf{Erfolgsszenario} &
            \begin{enumerate}
                \item{Der Betrachter startet die Player-Applikation und wählt
                        die gewünschten Optionen, wie Auflösung und Bit-Tiefe.}
                \item{Der Betrachter startet die Animation.}
                \item{Der Player spielt die Animation bis zum Ende.}
                \item{Der Player wird automatisch geschlossen.}
            \end{enumerate} \\
        \cmidrule(r){1-1}\cmidrule(lr){2-2}
            \textbf{Erweiterungen} &
            \begin{enumerate}[label= (\alph*)]
                \item{Vorzeitiger Abbruch durch den Betrachter
                    \begin{enumerate}[label= (\roman*)]
                            \item{Der Betrachter bricht eine laufende Animation
                                    ab.}
                            \item{Der Player wird sofort geschlossen.}
                    \end{enumerate}
                }
                \item{Absturz/Ausfall des Systems
                    \begin{enumerate}[label= (\roman*)]
                            \item{Das System stürzt an einem beliebigen Punkt
                                    ab.}
                            \item{Neustart des Players.}
                            \item{Der Player beginnt die Animation von vorne.}
                    \end{enumerate}
                }
            \end{enumerate}
            \\
        \cmidrule(r){1-1}\cmidrule(lr){2-2}
            \textbf{Zusätzliche Anforderungen} &
            \todo[inline]{Add add. requirements} \\
        \bottomrule
    \end{tabular}
\end{table}


% -*- coding: UTF-8 -*-
% vim: autoindent expandtab tabstop=4 sw=4 sts=4 filetype=tex
% vim: spelllang=de spell
% chktex-file 27 - disable warning about missing include files

\subsubsection{Use Case UC2: Erstellen einer Echtzeit-Animation}
\label{ssubsec:requirements:use-cases:uc2}

\begin{longtabu}{p{0.5\textwidth}p{0.5\textwidth}}
    \centering\\
    \caption{Use Case UC2: Erstellen einer
        Echtzeit-Animation.}\label{table:uc2-create-demo}\\
    \toprule
        \textbf{Bereich} &
        Editor \\
    \cmidrule(r){1-1}\cmidrule(lr){2-2}
        \textbf{Stufe (level)} &
        Ziel des Benutzers \\
    \cmidrule(r){1-1}\cmidrule(lr){2-2}
        \textbf{Primärer Aktor} &
        Anwender \\
    \cmidrule(r){1-1}\cmidrule(lr){2-2}
        \textbf{Stakeholder und Interessen} &
        Anwender: Möchte eine neue Echtzeit-Animation erstellen.\newline
        Betrachter: Möchte eine erstellte Echtzeit-Animation ansehen. \\
    \cmidrule(r){1-1}\cmidrule(lr){2-2}
        \textbf{Vorbedingungen} &
        Der Editor sichert eine Animation alle \textit{n}-Minuten in einer
        temporären Sicherungsdatei. \\
    \cmidrule(r){1-1}\cmidrule(lr){2-2}
        \textbf{Erfolgsszenario} &
        \begin{enumerate}
            \item{Der Anwender startet die Editor-Applikation.}
            \item{Der Anwender fügt Elemente zu einer Szene zusammen.}
            \item{Der Anwender legt Start und Ende einer Animation fest.}
            \item{Der Anwender speichert die getätigte Arbeit.}
            \item{Der Anwender exportiert die erstellte Animation.}
            \item{Der Anwender schliesst den Editor.}
        \end{enumerate} \\
    \cmidrule(r){1-1}\cmidrule(lr){2-2}
        \textbf{Erweiterungen} &
        \begin{enumerate}[label= (\alph*)]
            \item{Schliessen des Editors bei gemachten Änderungen ohne zu
                    speichern
                \begin{enumerate}[label= (\roman*)]
                    \item{Der Anwender schliesst den Editor bei gemachten
                            Änderungen ohne diese zu speichern.}
                    \item{Der Editor weist den Anwender auf die
                            nicht gespeicherten Änderungen hin und bietet
                            die Möglichkeit diese zu speichern, diese nicht
                            zu speichern oder das Schliessen abzubrechen.}
                \end{enumerate}
            }\\
        \end{enumerate}\\
    \cmidrule(r){1-1}\cmidrule(lr){2-2}
        \textbf{Fehlerfall} &
        \begin{enumerate}[label= (\alph*)]
            \item{Absturz/Ausfall des Systems
                \begin{enumerate}[label= (\roman*)]
                        \item{Das System stürzt an einem beliebigen Punkt
                                ab.}
                        \item{Neustart des Editors.}
                        \item{Der Editor stellt den ihm zuletzt bekannten
                                Punkt der Animation wieder her.}
                \end{enumerate}
            }
            \item{Es sind keine Elemente zum Hinzufügen vorhanden
                \begin{enumerate}[label= (\roman*)]
                    \item{Das Fenster zum Hinzufügen von Elementen ist leer.}
                \end{enumerate}
            }
            \item{Die getätigte Arbeit kann nicht gespeichert werden
                \begin{enumerate}[label= (\roman*)]
                    \item{Beim Speichern tritt ein Fehler auf.}
                    \item{Der Anwender wird via Dialog-Fenster über den Fehler
                            informiert.}
                    \item{Die Datei wird nicht gespeichert.}
                    \item{Die Datei gilt nach wie vor als geändert.}
                    \item{Die Datei muss zu einem späteren Zeitpunkt
                            gespeichert werden.}
                \end{enumerate}
            }
            \item{Die erstellte Animation kann nicht exportiert werden
                \begin{enumerate}[label= (\roman*)]
                    \item{Beim Exportieren tritt ein Fehler auf.}
                    \item{Der Anwender wird via Dialog-Fenster über den Fehler
                            informiert.}
                    \item{Die Datei wird nicht exportiert.}
                    \item{Die Datei muss zu einem späteren Zeitpunkt
                            exportiert werden.}
                \end{enumerate}
            }
        \end{enumerate} \\
    \cmidrule(r){1-1}\cmidrule(lr){2-2}
        \textbf{Zusätzliche Anforderungen} &
        Betreffend dem Exportieren ist das zu verwendende Format zum jetzigen
        Zeitpunkt noch unklar. In Frage käme zum Beispiel
        JSON\protect\footnotemark{} oder X3D\protect\footnotemark{}.\\
        \footnotetext{http://www.json.org/}
        \footnotetext{http://www.web3d.org/x3d}
    \bottomrule
\end{longtabu}

% -*- coding: UTF-8 -*-
% vim: autoindent expandtab tabstop=4 sw=4 sts=4 filetype=tex
% vim: spelllang=de spell
% chktex-file 27 - disable warning about missing include files

\begin{table}[H]
    \centering
    \caption{Use Case UC3: Bearbeiten einer bestehenden Animation}\label{table:uc3-edit-demo}
    \begin{tabular}{p{0.5\textwidth}p{0.5\textwidth}}
        \toprule
            \textbf{Bereich} &
            Editor \\
        \cmidrule(r){1-1}\cmidrule(lr){2-2}
            \textbf{Level} &
            Ziel des Benutzers \\
        \cmidrule(r){1-1}\cmidrule(lr){2-2}
            \textbf{Primärer Aktor} &
            Anwender \\
        \cmidrule(r){1-1}\cmidrule(lr){2-2}
            \textbf{Stakeholder und Interessen} &
            Anwender: Möchte eine zuvor erstellte Echtzeit-Animation verändern. \newline
            Betrachter: Möchte eine geänderte Echtzeit-Animation ansehen. \newline
            Entwickler: Testen einer geänderten Echtzeit-Animation. \\
        \cmidrule(r){1-1}\cmidrule(lr){2-2}
            \textbf{Erfolgsszenario} &
            \begin{enumerate}
                \item{Der Anwender startet die Editor-Applikation.}
                    \item{Der Anwender öffnet eine zuvor gespeicherte
                            Echtzeit-Animation.}
                \item{Der Editor lädt die gewählte Echtzeit-Animation.}
                \item{Der Anwender nimmt eine oder mehrere Änderungen vor.}
                \item{Der Anwender speichert die getätigte Arbeit.}
                \item{Der Anwender exportiert die erstellte Animation.}
                \item{Der Anwender schliesst den Editor.}
            \end{enumerate} \\
        \cmidrule(r){1-1}\cmidrule(lr){2-2}
            \textbf{Erweiterungen} &
            \begin{enumerate}[label= (\alph*)]
                \item{Schliessen des Editors bei gemachten Änderungen ohne zu
                        speichern
                    \begin{enumerate}[label= (\roman*)]
                        \item{Der Anwender schliesst den Editor bei gemachten
                                Änderungen ohne diese zu speichern.}
                        \item{Der Editor weist den Anwender auf die
                                nicht gespeicherten Änderungen hin und bietet
                                die Möglichkeit diese zu speichern, diese nicht
                                zu speichern oder das Schliessen abzubrechen.}
                    \end{enumerate}
                }
                \item{Eine zuvor gespeicherte Echtzeit-Animation kann nicht
                        geladen werden.
                    \begin{enumerate}[label= (\roman*)]
                        \item{Der Anwender öffnet eine zuvor gespeicherte
                                Echtzeit-Animation.}
                        \item{Der Editor weist den Anwender auf den Umstand
                                hin, dass die Echtzeit-Animation nicht geöffnet
                                werden kann.}
                    \end{enumerate}
                }
                \item{Benötigte zusätzliche Dateien können nicht gefunden
                        werden.
                    \begin{enumerate}[label= (\roman*)]
                        \item{Der Anwender öffnet eine zuvor gespeicherte
                                Echtzeit-Animation bei welcher benötigte
                                Dateien (wie zum Beispiel Bitmaps) fehlen.}
                        \item{Der Editor weist den Anwender auf den Umstand
                                hin, öffnet die Echtzeit-Animation dennoch. Er
                                ersetzt die fehlenden Dateien mit
                                Platzhaltern. Ein fehlerfreier Ablauf der
                                Echtzeit-Animation ist nicht gewährleistet.}
                    \end{enumerate}
                }
                \item{Absturz/Ausfall des Systems
                    \begin{enumerate}[label= (\roman*)]
                            \item{Das System stürzt an einem beliebigen Punkt
                                    ab.}
                            \item{Neustart des Editors.}
                            \item{Der Editor stellt den ihm zuletzt bekannten
                                    Punkt der Animation wieder her.}
                    \end{enumerate}
                }
            \end{enumerate}
            \\
        \cmidrule(r){1-1}\cmidrule(lr){2-2}
            \textbf{Zusätzliche Anforderungen} &
            \todo[inline]{Add add. requirements} \\
        \bottomrule
    \end{tabular}
\end{table}

% -*- coding: UTF-8 -*-
% vim: autoindent expandtab tabstop=4 sw=4 sts=4 filetype=tex
% vim: spelllang=de spell
% chktex-file 27 - disable warning about missing include files

\begin{table}[H]
    \centering
    \caption{Use Case UC4: Exportieren einer Animation.}\label{table:uc4-export-demo}
    \begin{tabular}{p{0.5\textwidth}p{0.5\textwidth}}
        \toprule
            \textbf{Bereich} &
            Editor \\
        \cmidrule(r){1-1}\cmidrule(lr){2-2}
            \textbf{Level} &
            Ziel des Benutzers \\
        \cmidrule(r){1-1}\cmidrule(lr){2-2}
            \textbf{Primärer Aktor} &
            Anwender \\
        \cmidrule(r){1-1}\cmidrule(lr){2-2}
            \textbf{Stakeholder und Interessen} &
            Anwender: Möchte eine erstellte Echtzeit-Animation für die
            Verwendung im Player exportieren. \newline
            Betrachter: Möchte eine Echtzeit-Animation ansehen. \newline
            Entwickler: Testen einer Echtzeit-Animation im Player. \\
        \cmidrule(r){1-1}\cmidrule(lr){2-2}
            \textbf{Erfolgsszenario} &
            \begin{enumerate}
                \item{Der Anwender startet die Editor-Applikation.}
                    \item{Der Anwender öffnet eine zuvor gespeicherte
                            Echtzeit-Animation oder erstellt eine neue
                            Echtzeit-Animation.}
                \item{Der Anwender speichert die getätigte Arbeit.}
                \item{Der Anwender exportiert die erstellte Animation.}
                \item{Der Editor exportiert die erstellte Animation mit allen
                        benötigen Abhängigkeiten in ein definiertes (Unter-)
                        Verzeichnis.}
                \item{Der Anwender schliesst den Editor.}
            \end{enumerate} \\
        \cmidrule(r){1-1}\cmidrule(lr){2-2}
            \textbf{Erweiterungen} &
            \begin{enumerate}[label= (\alph*)]
                \item{Schliessen des Editors bei gemachten Änderungen ohne zu
                        speichern
                    \begin{enumerate}[label= (\roman*)]
                        \item{Der Anwender schliesst den Editor bei gemachten
                                Änderungen ohne diese zu speichern.}
                        \item{Der Editor weist den Anwender auf die
                                nicht gespeicherten Änderungen hin und bietet
                                die Möglichkeit diese zu speichern, diese nicht
                                zu speichern oder das Schliessen abzubrechen.}
                    \end{enumerate}
                }
                \item{Eine zuvor gespeicherte Echtzeit-Animation kann nicht
                        geladen werden.
                    \begin{enumerate}[label= (\roman*)]
                        \item{Der Anwender öffnet eine zuvor gespeicherte
                                Echtzeit-Animation.}
                        \item{Der Editor weist den Anwender auf den Umstand
                                hin, dass die Echtzeit-Animation nicht geöffnet
                                werden kann.}
                    \end{enumerate}
                }
                \item{Benötigte zusätzliche Dateien können nicht gefunden
                        werden.
                    \begin{enumerate}[label= (\roman*)]
                        \item{Der Anwender öffnet eine zuvor gespeicherte bzw.
                                erstellt eine neue Echtzeit-Animation bei
                                welcher benötigte Dateien (wie zum Beispiel
                                Bitmaps) fehlen.}
                        \item{Der Editor weist den Anwender auf den Umstand
                                hin, exportiert die Echtzeit-Animation dennoch. Er
                                ersetzt die fehlenden Dateien mit
                                Platzhaltern. Ein fehlerfreier Ablauf der
                                Echtzeit-Animation ist nicht gewährleistet.}
                    \end{enumerate}
                }
                \item{Absturz/Ausfall des Systems
                    \begin{enumerate}[label= (\roman*)]
                            \item{Das System stürzt an einem beliebigen Punkt
                                    ab.}
                            \item{Neustart des Editors.}
                            \item{Der Editor stellt den ihm zuletzt bekannten
                                    Punkt der Animation wieder her.}
                    \end{enumerate}
                }
            \end{enumerate}
            \\
        \cmidrule(r){1-1}\cmidrule(lr){2-2}
            \textbf{Zusätzliche Anforderungen} &
            \todo[inline]{Add add. requirements} \\
        \bottomrule
    \end{tabular}
\end{table}

% -*- coding: UTF-8 -*-
% vim: autoindent expandtab tabstop=4 sw=4 sts=4 filetype=tex
% vim: spelllang=de spell
% chktex-file 27 - disable warning about missing include files

\begin{table}[H]
    \centering
    \caption{Use Case UC5: Bereitstellen neuer Editor-Elemente.}\label{table:uc1-user}
    \begin{tabular}{p{0.5\textwidth}p{0.5\textwidth}}
        \toprule
            \textbf{Bereich} &
            Player\\
        \cmidrule(r){1-1}\cmidrule(lr){2-2}
            \textbf{Level} &
            Ziel des Benutzers \\
        \cmidrule(r){1-1}\cmidrule(lr){2-2}
            \textbf{Primärer Aktor} &
            Entwickler \\
        \cmidrule(r){1-1}\cmidrule(lr){2-2}
            \textbf{Stakeholder und Interessen} &
            Entwickler: Möchte dem Anwender neue Möglichkeiten zur visuellen
            Gestaltung bieten. Möchte dem Betrachter neue, visuell ansprechende
            Elemente bieten. \newline
            Anwender: Möchte neue Funktionen zur Erstellung von
            Echtzeit-Animation nutzen können. \newline
            Betrachter: Möchte visuell ansprechende Echtzeit-Animationen mit
            neuen Effekten ansehen. \\
        \cmidrule(r){1-1}\cmidrule(lr){2-2}
            \textbf{Erfolgsszenario} &
            \begin{enumerate}
                \item{Der Entwickler entwickelt neuen Inhalt in Form eines
                        Knotens für den Graphen (dies kann zum Beispiel ein
                        prozedurales Mesh oder ein dedizierter Shader sein).}
                \item{Der Entwickler startet den Editor.}
                \item{Der neue Knoten steht im Editor automatisch zur Verfügung
                        und kann genutzt werden.}
                \item{Der Entwickler schliesst den Editor.}
            \end{enumerate} \\
        \cmidrule(r){1-1}\cmidrule(lr){2-2}
            \textbf{Erweiterungen} &
            \begin{enumerate}[label= (\alph*)]
                \item{Erfassen neuer Knoten direkt im Editor
                    \begin{enumerate}[label= (\roman*)]
                        \item{Der Entwickler startet den Editor.}
                        \item{Der Entwickler öffnet die Bibliothek mit allen
                                Knoten-Typen.}
                        \item{Der Entwickler fügt der Bibliothek einen neuen
                                Knoten hinzu.}
                        \item{Der Entwickler füllt den Knoten mit den
                                entsprechenden Details und speichert diesen.}
                        \item{Der neu erstellte Knoten wird persistiert.}
                        \item{Der neu erstellte Knoten erscheint in der
                                Bibliothek und ist per sofort auswählbar.}
                        \item{Der Entwickler schliesst den Editor.}
                    \end{enumerate}
                }
                \item{Fehlerhafter Inhalt des Knotens
                    \begin{enumerate}[label= (\roman*)]
                        \item{Der Entwickler fügt dem Knoten keinen oder
                                fehlerhaften Inhalt hinzu.}
                        \item{Der Editor weist den Entwickler auf den Umstand
                                hin. Der Knoten kann nicht verwendet werden.}
                    \end{enumerate}
                }
                \item{Absturz/Ausfall des Systems
                    \begin{enumerate}[label= (\roman*)]
                            \item{Das System stürzt an einem beliebigen Punkt
                                    ab.}
                            \item{Neustart des Editors.}
                            \item{Der Editor stellt den ihm zuletzt bekannten
                                    Punkt der Animation wieder her.}
                    \end{enumerate}
                }
            \end{enumerate}
            \\
        \cmidrule(r){1-1}\cmidrule(lr){2-2}
            \textbf{Zusätzliche Anforderungen} &
            \todo[inline]{Add add. requirements} \\
        \bottomrule
    \end{tabular}
\end{table}


