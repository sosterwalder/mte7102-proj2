% -*- coding: UTF-8 -*-
% vim: autoindent expandtab tabstop=4 sw=4 sts=4 filetype=tex
% vim: spelllang=de spell
% chktex-file 27 - disable warning about missing include files

\subsubsection{Use Case UC3: Bearbeiten einer bestehenden Animation}
\label{ssubsec:requirements:use-cases:uc3}

\begin{longtabu}{p{0.5\textwidth}p{0.5\textwidth}}
    \centering\\
    \caption{Use Case UC3: Bearbeiten einer bestehenden
        Animation}\label{table:uc3-edit-demo}\\
    \toprule
        \textbf{Bereich} &
        Editor \\
    \cmidrule(r){1-1}\cmidrule(lr){2-2}
        \textbf{Stufe (level)} &
        Ziel des Benutzers \\
    \cmidrule(r){1-1}\cmidrule(lr){2-2}
        \textbf{Primärer Aktor} &
        Anwender \\
    \cmidrule(r){1-1}\cmidrule(lr){2-2}
        \textbf{Stakeholder und Interessen} &
        Anwender: Möchte eine zuvor erstellte Echtzeit-Animation verändern. \newline
        Betrachter: Möchte eine geänderte Echtzeit-Animation ansehen. \newline
        Entwickler: Testen einer geänderten Echtzeit-Animation. \\
    \cmidrule(r){1-1}\cmidrule(lr){2-2}
        \textbf{Erfolgsszenario} &
        \begin{enumerate}
            \item{Der Anwender startet die Editor-Applikation.}
            \item{Der Anwender öffnet eine zuvor gespeicherte
                    Echtzeit-Animation.}
            \item{Der Editor lädt die gewählte Echtzeit-Animation.}
            \item{Der Anwender nimmt eine oder mehrere Änderungen vor.}
            \item{Der Anwender speichert die getätigte Arbeit.}
            \item{Der Anwender exportiert die erstellte Animation.}
            \item{Der Anwender schliesst den Editor.}
        \end{enumerate} \\
    \cmidrule(r){1-1}\cmidrule(lr){2-2}
        \textbf{Erweiterungen} &
        \begin{enumerate}[label= (\alph*)]
            \item{Schliessen des Editors bei gemachten Änderungen ohne zu
                    speichern
                \begin{enumerate}[label= (\roman*)]
                    \item{Der Anwender schliesst den Editor bei gemachten
                            Änderungen ohne diese zu speichern.}
                    \item{Der Editor weist den Anwender auf die
                            nicht gespeicherten Änderungen hin und bietet
                            die Möglichkeit diese zu speichern, diese nicht
                            zu speichern oder das Schliessen abzubrechen.}
                \end{enumerate}
            }
        \end{enumerate}\\
    \cmidrule(r){1-1}\cmidrule(lr){2-2}
        \textbf{Fehlerfall} &
        \begin{enumerate}[label= (\alph*)]
            \item{Eine zuvor gespeicherte Echtzeit-Animation kann nicht
                    geladen werden.
                \begin{enumerate}[label= (\roman*)]
                    \item{Der Anwender öffnet eine zuvor gespeicherte
                            Echtzeit-Animation.}
                    \item{Der Editor weist den Anwender auf den Umstand
                            hin, dass die Echtzeit-Animation nicht geöffnet
                            werden kann.}
                \end{enumerate}
            }
            \item{Benötigte zusätzliche Dateien können nicht gefunden
                    werden.
                \begin{enumerate}[label= (\roman*)]
                    \item{Der Anwender öffnet eine zuvor gespeicherte
                            Echtzeit-Animation bei welcher benötigte
                            Dateien (wie zum Beispiel Bitmaps) fehlen.}
                    \item{Der Editor weist den Anwender auf den Umstand
                            hin, öffnet die Echtzeit-Animation dennoch. Er
                            ersetzt die fehlenden Dateien mit
                            Platzhaltern. Ein fehlerfreier Ablauf der
                            Echtzeit-Animation ist nicht gewährleistet.}
                \end{enumerate}
            }
            \item{Absturz/Ausfall des Systems
                \begin{enumerate}[label= (\roman*)]
                        \item{Das System stürzt an einem beliebigen Punkt
                                ab.}
                        \item{Neustart des Editors.}
                        \item{Der Editor stellt den ihm zuletzt bekannten
                                Punkt der Animation wieder her.}
                \end{enumerate}
            }
            \item{Die getätigte Arbeit kann nicht gespeichert werden
                \begin{enumerate}[label= (\roman*)]
                    \item{Beim Speichern tritt ein Fehler auf.}
                    \item{Der Anwender wird via Dialog-Fenster über den Fehler
                            informiert.}
                    \item{Die Datei wird nicht gespeichert.}
                    \item{Die Datei gilt nach wie vor als geändert.}
                    \item{Die Datei muss zu einem späteren Zeitpunkt
                            gespeichert werden.}
                \end{enumerate}
            }
            \item{Die erstellte Animation kann nicht exportiert werden
                \begin{enumerate}[label= (\roman*)]
                    \item{Beim Exportieren tritt ein Fehler auf.}
                    \item{Der Anwender wird via Dialog-Fenster über den Fehler
                            informiert.}
                    \item{Die Datei wird nicht exportiert.}
                    \item{Die Datei muss zu einem späteren Zeitpunkt
                            exportiert werden.}
                \end{enumerate}
            }
        \end{enumerate} \\
    \cmidrule(r){1-1}\cmidrule(lr){2-2}
        \textbf{Zusätzliche Anforderungen} &
        ---\\
    \bottomrule
\end{longtabu}
