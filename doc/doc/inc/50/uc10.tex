% -*- coding: UTF-8 -*-
% vim: autoindent expandtab tabstop=4 sw=4 sts=4 filetype=tex
% vim: spelllang=de spell
% chktex-file 27 - disable warning about missing include files

\subsubsection{Use Case UC10: Auswerten und Darstellen eines Knotens}
\label{ssubsec:requirements:use-cases:uc10}

\begin{longtabu}{p{0.5\textwidth}p{0.5\textwidth}}
    \centering\\
    \caption{Use Case UC10: Auswerten und Darstellen eines
        Knotens.}\label{table:uc10-render-node}\\
    \toprule
        \textbf{Bereich} &
        Editor und Player \\
    \cmidrule(r){1-1}\cmidrule(lr){2-2}
        \textbf{Stufe (level)} &
        Ziel des Benutzers und Unterfunktion\\
    \cmidrule(r){1-1}\cmidrule(lr){2-2}
        \textbf{Primärer Aktor} &
        Anwender \\
    \cmidrule(r){1-1}\cmidrule(lr){2-2}
        \textbf{Stakeholder und Interessen} &
        Anwender: Möchte das (visuelle) Ergebnis eines Knotens
        sehen.\newline
        Betrachter: Möchte animierten Inhalt sehen.\newline
        Entwickler: Testen von Knoten und Unterknoten. \\
    \cmidrule(r){1-1}\cmidrule(lr){2-2}
        \textbf{Vorbedingungen Editor} &
        \begin{itemize}
            \item{Die Editor-Applikation ist gestartet.}
            \item{Es sind Knoten-Typen zum Hinzufügen vorhanden.}
            \item{Eine beliebige Szene ist angewählt.}
            \item{Die Szene verfügt bereits über Knoten.}
        \end{itemize} \\
    \cmidrule(r){1-1}\cmidrule(lr){2-2}
        \textbf{Vorbedingungen Player} &
        \begin{itemize}
            \item{Die Player-Applikation ist gestartet.}
            \item{Es ist ein Demo-Skript mitsamt Inhalt zum Ausführen
                    vorhanden.}
            \item{Das Demo-Skript verfügt über mindestens eine Szene.}
            \item{Die Szene verfügt über Knoten.}
            \item{Die Szene ist mit dem Haupt-Ausgabeknoten verbunden.}
        \end{itemize} \\
    \cmidrule(r){1-1}\cmidrule(lr){2-2}
    \textbf{Erfolgsszenario Editor} &
        \begin{enumerate}
            \item{Der Anwender klickt mit der linken Maustaste auf den
                    Knoten, dessen Ausgabe er sehen möchte.}
            \item{Der Knoten wird im Graphen als markiert dargestellt.}
            \item{Die Parameter des Knoten werden im Parameter-Fenster
                    dargestellt.}
            \item{Der Editor evaluiert den Graphen des Knoten rekursiv zum
                    aktuellen Zeitpunkt der Zeitachse.}
            \item{Der Editor stellt die berechnete Ausgabe im
                    Rendering-Ansichtsfenster dar.}
        \end{enumerate} \\
    \cmidrule(r){1-1}\cmidrule(lr){2-2}
    \textbf{Erfolgsszenario Player} &
        \begin{enumerate}
            \item{Der Player lädt das Demoskript mitsamt all seinen
                    Ressourcen und evaluiert. Dabei wird zuerst der
                    Haupt-Ausgabeknoten evaluiert und dann rekursiv
                    traversiert.}
            \item{Ist eine Animation vorhanden, so wird diese nun
                    gestartet und der Graph des Haupt-Ausgabeknotens rekursiv zum
                    aktuellen Zeitpunkt der Animation evaluiert.}
            \item{Ist keine Animation vorhanden, so wird immer der
                    Zeitpunkt 0 evaluiert.}
            \item{Der Player stellt die berechnete Ausgabe im
                    Rendering-Ansichtsfenster dar.}
        \end{enumerate} \\
    \cmidrule(r){1-1}\cmidrule(lr){2-2}
        \textbf{Erweiterungen} &
        \begin{enumerate}[label= (\alph*)]
            \item{Absturz/Ausfall des Systems
                \begin{enumerate}[label= (\roman*)]
                        \item{Das System stürzt an einem beliebigen Punkt
                                ab.}
                        \item{Neustart des Editors/Players.}
                        \item{Der Editor stellt den ihm zuletzt bekannten
                                Punkt der Animation wieder her. Der Player
                            beginnt die Animation von vorne --- sofern eine
                        solche vorhanden ist.}
                \end{enumerate}
            } \end{enumerate}
        \\
    \cmidrule(r){1-1}\cmidrule(lr){2-2}
        \textbf{Zusätzliche Anforderungen} &
        --- \\
    \bottomrule
\end{longtabu}
