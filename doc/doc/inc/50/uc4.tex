% -*- coding: UTF-8 -*-
% vim: autoindent expandtab tabstop=4 sw=4 sts=4 filetype=tex
% vim: spelllang=de spell
% chktex-file 27 - disable warning about missing include files

\begin{table}[H]
    \centering
    \caption{Use Case UC4: Exportieren einer Animation.}\label{table:uc4-export-demo}
    \begin{tabular}{p{0.5\textwidth}p{0.5\textwidth}}
        \toprule
            \textbf{Bereich} &
            Editor \\
        \cmidrule(r){1-1}\cmidrule(lr){2-2}
            \textbf{Level} &
            Ziel des Benutzers \\
        \cmidrule(r){1-1}\cmidrule(lr){2-2}
            \textbf{Primärer Aktor} &
            Anwender \\
        \cmidrule(r){1-1}\cmidrule(lr){2-2}
            \textbf{Stakeholder und Interessen} &
            Anwender: Möchte eine erstellte Echtzeit-Animation für die
            Verwendung im Player exportieren. \newline
            Betrachter: Möchte eine Echtzeit-Animation ansehen. \newline
            Entwickler: Testen einer Echtzeit-Animation im Player. \\
        \cmidrule(r){1-1}\cmidrule(lr){2-2}
            \textbf{Erfolgsszenario} &
            \begin{enumerate}
                \item{Der Anwender startet die Editor-Applikation.}
                    \item{Der Anwender öffnet eine zuvor gespeicherte
                            Echtzeit-Animation oder erstellt eine neue
                            Echtzeit-Animation.}
                \item{Der Anwender speichert die getätigte Arbeit.}
                \item{Der Anwender exportiert die erstellte Animation.}
                \item{Der Editor exportiert die erstellte Animation mit allen
                        benötigen Abhängigkeiten in ein definiertes (Unter-)
                        Verzeichnis.}
                \item{Der Anwender schliesst den Editor.}
            \end{enumerate} \\
        \cmidrule(r){1-1}\cmidrule(lr){2-2}
            \textbf{Erweiterungen} &
            \begin{enumerate}[label= (\alph*)]
                \item{Schliessen des Editors bei gemachten Änderungen ohne zu
                        speichern
                    \begin{enumerate}[label= (\roman*)]
                        \item{Der Anwender schliesst den Editor bei gemachten
                                Änderungen ohne diese zu speichern.}
                        \item{Der Editor weist den Anwender auf die
                                nicht gespeicherten Änderungen hin und bietet
                                die Möglichkeit diese zu speichern, diese nicht
                                zu speichern oder das Schliessen abzubrechen.}
                    \end{enumerate}
                }
                \item{Eine zuvor gespeicherte Echtzeit-Animation kann nicht
                        geladen werden.
                    \begin{enumerate}[label= (\roman*)]
                        \item{Der Anwender öffnet eine zuvor gespeicherte
                                Echtzeit-Animation.}
                        \item{Der Editor weist den Anwender auf den Umstand
                                hin, dass die Echtzeit-Animation nicht geöffnet
                                werden kann.}
                    \end{enumerate}
                }
                \item{Benötigte zusätzliche Dateien können nicht gefunden
                        werden.
                    \begin{enumerate}[label= (\roman*)]
                        \item{Der Anwender öffnet eine zuvor gespeicherte bzw.
                                erstellt eine neue Echtzeit-Animation bei
                                welcher benötigte Dateien (wie zum Beispiel
                                Bitmaps) fehlen.}
                        \item{Der Editor weist den Anwender auf den Umstand
                                hin, exportiert die Echtzeit-Animation dennoch. Er
                                ersetzt die fehlenden Dateien mit
                                Platzhaltern. Ein fehlerfreier Ablauf der
                                Echtzeit-Animation ist nicht gewährleistet.}
                    \end{enumerate}
                }
                \item{Absturz/Ausfall des Systems
                    \begin{enumerate}[label= (\roman*)]
                            \item{Das System stürzt an einem beliebigen Punkt
                                    ab.}
                            \item{Neustart des Editors.}
                            \item{Der Editor stellt den ihm zuletzt bekannten
                                    Punkt der Animation wieder her.}
                    \end{enumerate}
                }
            \end{enumerate}
            \\
        \cmidrule(r){1-1}\cmidrule(lr){2-2}
            \textbf{Zusätzliche Anforderungen} &
            \todo[inline]{Add add. requirements} \\
        \bottomrule
    \end{tabular}
\end{table}
