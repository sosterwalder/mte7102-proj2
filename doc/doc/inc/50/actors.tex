% -*- coding: UTF-8 -*-
% vim: autoindent expandtab tabstop=4 sw=4 sts=4 filetype=tex
% vim: spelllang=de spell
% chktex-file 27 - disable warning about missing include files

\subsection{Akteure (actors)}
\label{subsec:requirements:actors}

Ein Akteur ist ein Objekt mit einem Verhalten (informal gesprochen), wie zum
Beispiel eine Person (definiert durch eine Rolle), ein Computersystem oder eine
Organisation~\cite[S.63]{larman_applying_2004}.

Nachfolgend finden sich alle Akteure der hier spezifizierten
Software-Architektur. Dabei wird zwischen primären und sekundären Akteuren
unterschieden.

\subsubsection{Primäre Akteure}
\label{ssubsec:requirements:actors:primary}

\begin{itemize}
    \item{%
            \textbf{Anwender}\\
            Ein \textit{Anwender} erstellt Echtzeit-Animationen mit der
            Software. Er hat also eine aktive Rolle.
        }
    \item{%
            \textbf{Betrachter}\\
            Ein \textit{Betrachter} nutzt die Software um eine durch den
            \textit{Anwender} erstellte Echtzeit-Animation anzusehen. Er nimmt
            also eine passive Rolle ein.
        }
\end{itemize}

\subsubsection{Sekundäre Akteure}
\label{ssubsec:requirements:actors:secondary}

\begin{itemize}
    \item{%
            \textbf{Entwickler}\\
            Ein \textit{Entwickler} erweitert die Software um neue Elemente
            (wie z.B.\ neue Shader).
        }
\end{itemize}

