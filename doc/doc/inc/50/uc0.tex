% -*- coding: UTF-8 -*-
% vim: autoindent expandtab tabstop=4 sw=4 sts=4 filetype=tex
% vim: spelllang=de spell
% chktex-file 27 - disable warning about missing include files

\begin{longtabu}{p{0.3\textwidth}p{0.7\textwidth}}
    \centering\\
    \caption{Erklärung der Begrifflichkeiten der Use
        Cases, angelehnt an~\cite[S.
        67]{larman_applying_2004}.}\label{table:uc-explanation}\\
    \toprule
        \textbf{Bereich} &
        Der Bereich des Use Cases. Dies ist entweder der Editor oder aber der
        Player (siehe hierzu auch~\autoref{subsec:requirements:vision}).\\
    \cmidrule(r){1-1}\cmidrule(lr){2-2}
        \textbf{Stufe (level)} &
        ``Ziel des Benutzers'' oder ``Unterfunktion''.\\
    \cmidrule(r){1-1}\cmidrule(lr){2-2}
        \textbf{Primärer Aktor} &
        Primärer Aktor des Use Cases: Anwender, Betrachter oder Entwickler.
        Siehe auch~\autoref{subsec:requirements:actors}. \\
    \cmidrule(r){1-1}\cmidrule(lr){2-2}
        \textbf{Stakeholder und Interessen} &
        Interessenten des Use Cases und deren Ziele.\\
    \cmidrule(r){1-1}\cmidrule(lr){2-2}
        \textbf{Erfolgsszenario} &
        Typisches, gewünschtes Szenario des Uses Cases, bei welchem keinerlei
        Fehler oder Nebenwirkungen auftreten.\\
    \cmidrule(r){1-1}\cmidrule(lr){2-2}
        \textbf{Fehlerfall} &
        Zusätzliche, fehlerhafte Szenarien.\\
    \cmidrule(r){1-1}\cmidrule(lr){2-2}
        \textbf{Erweiterungen} &
        Zusätzliche Szenarien.\\
    \cmidrule(r){1-1}\cmidrule(lr){2-2}
        \textbf{Zusätzliche Anforderungen} &
        Non-funktionelle Anforderungen, welche in Relation zu dem Use Case
        stehen.\\
    \bottomrule
\end{longtabu}
