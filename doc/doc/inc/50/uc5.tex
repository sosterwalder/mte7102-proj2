% -*- coding: UTF-8 -*-
% vim: autoindent expandtab tabstop=4 sw=4 sts=4 filetype=tex
% vim: spelllang=de spell
% chktex-file 27 - disable warning about missing include files

\subsubsection{Use Case UC5: Bereitstellen neuer Editor-Elemente}
\label{ssubsec:requirements:use-cases:uc5}

\begin{longtabu}{p{0.5\textwidth}p{0.5\textwidth}}
    \centering\\
    \caption{Use Case UC5: Bereitstellen neuer
        Editor-Elemente.}\label{table:uc5-add-new-elements}\\
    \toprule
        \textbf{Bereich} &
        Player\\
    \cmidrule(r){1-1}\cmidrule(lr){2-2}
        \textbf{Stufe (level)} &
        Ziel des Benutzers \\
    \cmidrule(r){1-1}\cmidrule(lr){2-2}
        \textbf{Primärer Aktor} &
        Entwickler \\
    \cmidrule(r){1-1}\cmidrule(lr){2-2}
        \textbf{Stakeholder und Interessen} &
        Entwickler: Möchte dem Anwender neue Möglichkeiten zur visuellen
        Gestaltung bieten. Möchte dem Betrachter neue, visuell ansprechende
        Elemente bieten. \newline
        Anwender: Möchte neue Funktionen zur Erstellung von
        Echtzeit-Animation nutzen können. \newline
        Betrachter: Möchte visuell ansprechende Echtzeit-Animationen mit
        neuen Effekten ansehen. \\
    \cmidrule(r){1-1}\cmidrule(lr){2-2}
        \textbf{Erfolgsszenario} &
        \begin{enumerate}
            \item{Der Entwickler entwickelt neuen Inhalt in Form eines
                    Knotens für den Graphen (dies kann zum Beispiel ein
                    prozedurales Mesh oder ein dedizierter Shader sein).}
            \item{Der Entwickler startet den Editor.}
            \item{Der neue Knoten steht im Editor automatisch zur Verfügung
                    und kann genutzt werden.}
            \item{Der Entwickler schliesst den Editor.}
        \end{enumerate} \\
    \cmidrule(r){1-1}\cmidrule(lr){2-2}
        \textbf{Erweiterungen} &
        \begin{enumerate}[label= (\alph*)]
            \item{Erfassen neuer Knoten direkt im Editor
                \begin{enumerate}[label= (\roman*)]
                    \item{Der Entwickler startet den Editor.}
                    \item{Der Entwickler öffnet die Bibliothek mit allen
                            Knoten-Typen.}
                    \item{Der Entwickler fügt der Bibliothek einen neuen
                            Knoten hinzu.}
                    \item{Der Entwickler füllt den Knoten mit den
                            entsprechenden Details und speichert diesen.}
                    \item{Der neu erstellte Knoten wird persistiert.}
                    \item{Der neu erstellte Knoten erscheint in der
                            Bibliothek und ist per sofort auswählbar.}
                    \item{Der Entwickler schliesst den Editor.}
                \end{enumerate}
            }
        \end{enumerate}\\
    \cmidrule(r){1-1}\cmidrule(lr){2-2}
        \textbf{Fehlerfall} &
        \begin{enumerate}[label= (\alph*)]
            \item{Fehlerhafter Inhalt des Knotens
                \begin{enumerate}[label= (\roman*)]
                    \item{Der Entwickler fügt dem Knoten keinen oder
                            fehlerhaften Inhalt hinzu.}
                    \item{Der Editor weist den Entwickler auf den Umstand
                            hin. Der Knoten kann nicht verwendet werden.}
                \end{enumerate}
            }
            \item{Absturz/Ausfall des Systems
                \begin{enumerate}[label= (\roman*)]
                        \item{Das System stürzt an einem beliebigen Punkt
                                ab.}
                        \item{Neustart des Editors.}
                        \item{Der Editor stellt den ihm zuletzt bekannten
                                Punkt der Animation wieder her.}
                \end{enumerate}
            }
            \item{Der Knoten kann nicht persistiert werden
                    \begin{enumerate}[label= (\roman*)]
                        \item{Beim Speichern tritt ein Fehler auf.}
                        \item{Der Entwickler wird via Dialog-Fenster über den Fehler
                                informiert.}
                        \item{Die Datei wird nicht gespeichert.}
                        \item{Die Datei gilt nach wie vor als nicht persistiert.}
                        \item{Die Datei muss zu einem späteren Zeitpunkt
                                gespeichert werden.}
                    \end{enumerate}
                }
        \end{enumerate}
        \\
    \cmidrule(r){1-1}\cmidrule(lr){2-2}
        \textbf{Zusätzliche Anforderungen} &
        ---\\
    \bottomrule
\end{longtabu}

