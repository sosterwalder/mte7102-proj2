% -*- coding: UTF-8 -*-
% vim: autoindent expandtab tabstop=4 sw=4 sts=4 filetype=tex
% vim: spelllang=de spell
% chktex-file 27 - disable warning about missing include files

\subsection{Zusätzliche Anforderungen}
\label{subsec:requirements:additional-requirements}

\subsubsection{Software}
\label{subsec:requirements:additional-requirements:software}

Durch die vorhergehende Projektarbeit --- MTE7101
(siehe~\cite{osterwalder_sven_volume_2016}) --- sowie persönlicher Erfahrungen
bei diversen Projekten, sieht der Autor die Software gemäss
Tabelle~\ref{table:requirements:additional-requirements:software} zur Umsetzung
vor. Alle genannten Komponenten --- ausser Qt --- beziehen sich auf den
\textit{Player}, als auch auf den \textit{Editor}.  Der Player benötigt kein
Qt, da er kein Frontend hat und möglichst schlank gehalten werden soll.

% Sigh..
% FYI:  Tabular and a fixed text-width of 79 sucks!
% FYI2: set cursorcolumn with a lot of text sucks too!

\begin{longtabu}{p{0.2\textwidth}p{0.7\textwidth}p{0.1\textwidth}}
    \caption{Mögliche
        Software/Technologien}\label{table:requirements:additional-requirements:software}\\
    \toprule
    \textbf{Komponente} & \textbf{Beschreibung} & \textbf{Verweise} \\
    \midrule
    C++        & Objektorientierte Programmiersprache
                           & \protect\footnotemark\footnotetext{
                               \url{http://www.iso.org/iso/iso_catalogue/catalogue_tc/catalogue_detail.htm?csnumber=50372}
                           }\\

    OpenGL     & Plattformunabhängige Programmierschnittstelle zur
                           Entwicklung von 2D- und 3D-Computergrafikanwendungen
                           \parencite{wikipedia_the_free_encyclopedia_opengl_2015}
                           &\protect\footnotemark\footnotetext{
                               \url{https://www.opengl.org/registry/doc/glspec45.core.pdf}
                           }\\

    Qt        & ``Qt ist eine C++-Klassenbibliothek für die
                           plattformübergreifende Programmierung grafischer
                           Benutzeroberflächen. Ausserdem bietet Qt
                           umfangreiche Funktionen zur Internationalisierung
                           sowie Datenbankfunktionen und XML-Unterstützung an
                           und ist für eine grosse Zahl an Betriebssystemen bzw.
                           Grafikplattformen, wie X11 (Unix-Derivate), OS X,
                           Windows, iOS und Android
                           erhältlich.''~\parencite{wikipedia_foundation_qt_2016}.
                           &\protect\footnotemark\footnotetext{
                               \url{https://www.qt.io}
                           }\\

    GLFW       & OpenGL-Bibliothek, welche die Erstellung und
                           Verwaltung von Fenstern sowie OpenGL-Kontexte
                           vereinfacht
                           \parencite{wikipedia_the_free_encyclopedia_glfw_2015}.
                           &\protect\footnotemark\footnotetext{
                               \url{http://www.glfw.org}
                           }\\

    GLEW       & OpenGL Extension Wrangler. Bibliothek zum Abfragen
                           und Laden von OpenGL-Erweiterungen (Extensions)
                           \parencite{wikipedia_the_free_encyclopedia_opengl_2015-1}.
                           &\protect\footnotemark\footnotetext{
                               \url{http://glew.sourceforge.net}
                           }\\

    GLM        & Header-only C++ Mathematik-Bibliothek, basierend auf
                           der Spezifikation der OpenGL Shader-Sprache GLSL.
                           Sie bietet eine Vielzahl an Datentypen wie etwa
                           Vektoren, Matrizen oder Quaternionen.
                           &\protect\footnotemark\footnotetext{
                               \url{https://glm.g-truc.net}
                           }\\

    CMake      & Software zur Verwaltung von Build-Prozessen von 
                           Software
                           &\protect\footnotemark\footnotetext{
                               \url{https://www.cmake.org}
                           }\\

    LLVM       & Ansammlung von modularen und wiederverwendbaren
                           Compilern und Toolchains.
                           &\protect\footnotemark\footnotetext{
                               \url{http://llvm.org}
                           }\\

    Clang      & Compiler-Frontend für LLVM.\@
                           &\protect\footnotemark\footnotetext{
                               \url{http://clang.llvm.org}
                           }\\

    Boost      & Freie Bibliothek bestehend aus einer Vielzahl von
                           Bibliotheken, die den unterschiedlichsten
                           Aufgaben von Algorithmen auf Graphen über 
                           Metaprogrammierung bis hin zu Speicherverwaltung
                           dienen
                           \parencite{wikipedia_the_free_encyclopedia_boost_2015}.
                           &\protect\footnotemark\footnotetext{
                               \url{http://www.boost.org}
                           }\\
    \bottomrule
\end{longtabu}
