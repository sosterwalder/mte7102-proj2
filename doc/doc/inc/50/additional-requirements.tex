% -*- coding: UTF-8 -*-
% vim: autoindent expandtab tabstop=4 sw=4 sts=4 filetype=tex
% vim: spelllang=de spell
% chktex-file 27 - disable warning about missing include files

\subsection{Zusätzliche Anforderungen}
\label{subsec::requirements:additional-requirements}

\subsubsection{Software}
\label{subsec::requirements:additional-requirements:software}

Durch die vorhergehende Projektarbeit --- MTE7101
(siehe~\cite{osterwalder_sven_volume_2016}) --- sowie persönlicher Erfahrungen
bei diversen Projekten, sieht der Autor die Software in der
nachfolgenden Tabelle zur Umsetzung vor. Alle genannten Komponenten --- ausser
Qt --- beziehen sich auf den \textit{Player}, als auch auf den \textit{Editor}.
Der Player benötigt kein Qt, da er kein Frontend hat und möglichst schlank
gehalten werden soll.

\begin{tabular}{llp{10cm}l}
    \toprule
    \textbf{Komponente} & \textbf{Version} & \textbf{Beschreibung} & \textbf{Verweise} \\
    \midrule
    C++        & 11      & Objektorientierte Programmiersprache
                           &\protect\footnotemark\\

    OpenGL     & 4.5     & Plattformunabhängige Programmierschnittstelle zur
                           Entwicklung von 2D- und 3D-Computergrafikanwendungen
                           \parencite{wikipedia_the_free_encyclopedia_opengl_2015}.
                           &\protect\footnotemark\\

    Qt        & 5.7      & ```Qt ist eine C++-Klassenbibliothek für die
                           plattformübergreifende Programmierung grafischer
                           Benutzeroberflächen.  Außerdem bietet Qt
                           umfangreiche Funktionen zur Internationalisierung
                           sowie Datenbankfunktionen und XML-Unterstützung an
                           und ist für eine große Zahl an Betriebssystemen bzw.
                           Grafikplattformen, wie X11 (Unix-Derivate), OS X,
                           Windows, iOS und Android
                           erhältlich.'''~\parencite{wikipedia_foundation_qt_2016}.
                           &\protect\footnotemark\\

    GLFW       & 3.1.2   & OpenGL-Bibliothek, welche die Erstellung und
                           Verwaltung von Fenstern sowie OpenGL-Kontexte
                           vereinfacht
                           \parencite{wikipedia_the_free_encyclopedia_glfw_2015}.
                           &\protect\footnotemark\\

    GLEW       & 1.13    & OpenGL Extension Wrangler. Bibliothek zum Abfragen
                           und Laden von OpenGL-Erweiterungen (Extensions)
                           \parencite{wikipedia_the_free_encyclopedia_opengl_2015-1}.
                           &\protect\footnotemark\\

    GLM        & 0.9.7.6 & Header-only C++ Mathematik-Bibliothek, basierend auf
                           der Spezifikation der OpenGL Shader-Sprache GLSL.
                           Sie bietet eine Vielzahl an Datentypen wie etwa
                           Vektoren, Matrizen oder Quaternionen.
                           &\protect\footnotemark\\

    CMake      & 3.3.2   & Software zur Verwaltung von Build-Prozessen von 
                           Software
                           &\protect\footnotemark\\

    LLVM       & 3.8.1   & Ansammlung von modularen und wiederverwendbaren
                           Compilern und Toolchains.
                           &\protect\footnotemark\\

    Clang      & 3.8.1   & Compiler-Frontend für LLVM.\@
                           &\protect\footnotemark\\

    Boost      & 1.59.0  & Freie Bibliothek bestehend aus einer Vielzahl von
                           Bibliotheken, die den unterschiedlichsten
                           Aufgaben von Algorithmen auf Graphen über 
                           Metaprogrammierung bis hin zu Speicherverwaltung
                           dienen
                           \parencite{wikipedia_the_free_encyclopedia_boost_2015}.
                           &\protect\footnotemark\\
    \bottomrule
\end{tabular}
\footnotetext{\url{http://www.iso.org/iso/iso_catalogue/catalogue_tc/catalogue_detail.htm?csnumber=50372}}
\footnotetext{\url{https://www.opengl.org/registry/doc/glspec45.core.pdf}}
\footnotetext{\url{https://www.qt.io}}
\footnotetext{\url{http://www.glfw.org}}
\footnotetext{\url{http://glew.sourceforge.net}}
\footnotetext{\url{https://glm.g-truc.net}}
\footnotetext{\url{https://www.cmake.org}}
\footnotetext{\url{http://llvm.org}}
\footnotetext{\url{http://clang.llvm.org}}
\footnotetext{\url{http://www.boost.org}}


