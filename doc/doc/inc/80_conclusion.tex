% -*- coding: UTF-8 -*-
% vim: autoindent expandtab tabstop=4 sw=4 sts=4 filetype=tex
% vim: spelllang=de spell
% chktex-file 27 - disable warning about missing include files

\chapter{Schlusswort}
\label{chap:discussion_and_conclusion}

In dieser Projektarbeit wurde Sphere Tracing vorgestellt. Es ist eine
optimierte Variante des als Ray Tracing bekannten Verfahrens.  Sphere
Tracing erlaubt die Darstellung von Szenen in der Qualität von Ray
Tracing, aber in Echtzeit.

Zuerst wurden \textit{lokale}, dann \textit{globale Beleuchtungsmodelle} inklusive der
\textit{Renderinggleichung} vorgestellt. Weiter wurde das \textit{Ray Tracing}
Verfahren ausgehend von der ursprünglichen Idee des Ray Tracings, ein
als \textit{Ray Casting} bekanntes Verfahren, vorgestellt. Dabei wurde auch die
zugrundeliegende Physik vereinfacht aufgezeigt.

Für die praktische Anwendung der Beleuchtungsmodelle wurden die
klassischen Modelle zur Schattierung \textit{Flat-Shading},
\textit{Gouraud-Shading} sowie \textit{Phong-Shading} vorgestellt.

Um die eigentliche Szene darstellen zu können,
wurden~\textit{Oberflächen} im Allgemeinen eingeführt. Dabei wurde
gezeigt, dass zur Modellierung von Oberflächen hauptsächlich zwei
Techniken verwendet werden: Die parametrische und die implizite
Modellierung bzw.\ Darstellung. Es wurden dann \textit{implizite
    Oberflächen} im Speziellen behandelt. Mittels
\textit{Distanzfunktionen} wurde eine Grundlage zur Berechnung von
Distanzen gezeigt, welche der späteren Modellierung und Darstellung
dient. Ausgehend von den Distanzfunktionen wurden schliesslich
\textit{Distanzfelder} beschrieben, eine Art Datenstruktur basierend auf
den Distanzfunktionen.

Mit \textit{Ray Marching} und \textit{Sphere Tracing} wurden zwei
Methoden zur Darstellung von impliziten Oberflächen aufgezeigt. Weiter
wurden \textit{Operationen für implizite Oberflächen}, wie
beispielsweise die Vereinigung oder Subtraktion, sowie Funktionen zur
Modellierung von \textit{geometrischen Primitiven} vorgestellt.

Schliesslich wurde gezeigt, wie die vorgestellten Grundlagen für das
Rendering von impliziten Oberflächen genutzt werden können.

Im letzten Abschnitt wurde der
umgesetzte~\hyperref[chap:prototype]{\textit{Prototyp}} vorgestellt,
wobei einzelne Parameter des~\hyperref[chap:prototype]{Prototyps}
(\textit{Distanz}, \textit{Präzision}, die \textit{Anzahl Schritte} und
der \textit{Skalierungsfaktor für Schatten}) anhand einer Beispielszene
verglichen wurden.

% -*- coding: UTF-8 -*-
% vim: autoindent expandtab tabstop=4 sw=4 sts=4 filetype=tex
% vim: spelllang=de spell
% chktex-file 27 - disable warning about missing include files

\section{Erweiterungsmöglichkeiten}
\label{sec:further_work}

Da diese Projektarbeit den begrenzten Zeitrahmen eines Semesters
hat, konnten nicht alle von mir gewünschten Themata bearbeitet werden.

Bei Sphere Tracing handelt es sich um ein relativ junges
Verfahren, welches ein grosses Potential für Forschung und
Entwicklung enthält. Angesichts des grossen Themengebietes kann man viel
Zeit investieren.

Während der Recherche sind einige Thematiken hinzugekommen, sie werden
aber nicht bearbeitet. Sie zeigen auf, in welche Richtung die
Weiterführung von Projektarbeit und~\hyperref[chap:prototype]{Prototyp} gehen kann.

\subsection{Umgebungs-Verdeckung (Ambient Occlusion)}
\label{subsec:further_work:ambient_occlusion}

``Umgebungsverdeckung (englisch Ambient Occlusion, AO) ist eine
Shading-Methode, die in der 3D-Computergrafik verwendet wird, um mit
relativ kurzer Renderzeit eine realistische Verschattung von Szenen zu
erreichen. Das Ergebnis ist zwar nicht physikalisch korrekt, reicht
jedoch in seinem Realismus oft aus, um auf rechenintensive globale
Beleuchtung verzichten zu
können.''~\parencite{wikipedia_the_free_encyclopedia_umgebungsverdeckung_2015}

\citeauthor{evans_fast_2006} liefert in seiner
Arbeit~\citetitle{evans_fast_2006} interessante Ansätze, wie die
Umgebungsverdeckung auf Distanzfelder angewendet werden kann. Dies kann
mit relativ wenig Aufwand auch für Sphere Tracing angewendet
werden~\parencite{evans_fast_2006}.

\subsection{Kantenglättung}
\label{subsec:further_work:antialiasing}

Bei der Darstellung von Szenen mittels Sphere Tracing kommt es an Kanten
zu harten Übergängen und Artefakten --- es tritt das so genannte
``Aliasing'' auf.

\citeauthor{hart_sphere_1994} schlägt in~\citetitle{hart_sphere_1994}
bereits Verfahren zur Kantenglättung vor. Diese verkomplizieren jedoch
den Rendering-Prozess und haben Schwächen im Umgang mit angenäherten
Obergrenzen von impliziten Oberflächen~\parencite{hart_sphere_1994}.

\citeauthor{lottes_fxaa_2009} schlägt in seiner
Arbeit~\citetitle{lottes_fxaa_2009} eine einfache Lösung dieser
Problematiken vor. Es handelt sich um einen Bild-Filter, welcher
typischerweise in der Nachbearbeitung eines Bildes zum Einsatz kommt.
Der Filter wird auf das gerenderte Bild (bzw.\ auf einem Bild-Puffer)
angewendet. Es handelt sich um ein Verfahren, welches auf der
Detektion von Kanten basiert~\parencite{lottes_fxaa_2009}.

\subsection{Realistische Materialien}
\label{subsec:further_work:brdf}

In~\autoref{subsec:ray_tracing} werden Verfahren zur Berechnung der
physikalischen Eigenschaften von Oberflächen und damit von Materialien
aufgezeigt. Dabei handelt es sich aber nur um Näherungen unter Annahme
von perfekten Bedingungen, was von der Realität weit entfernt ist.

Eine allgemeinere und daher realistischerer Form ist die so genannte
BRDF:\ ``Eine bidirektionale Reflektanzverteilungsfunktion (engl.
Bidirectional Reflectance Distribution Function, BRDF) stellt eine
Funktion für das Reflexionsverhalten von Oberflächen eines Materials
unter beliebigen Einfallswinkeln dar. Sie liefert für jeden auf dem
Material auftreffenden Lichtstrahl mit gegebenem Eintrittswinkel den
Quotienten aus Strahlungsdichte und Bestrahlungsstärke für jeden
austretenden Lichtstrahl. BRDFs werden unter anderem in der
realistischen 3D-Computergrafik verwendet, wo sie einen Teil der
fundamentalen Rendergleichung darstellen und dazu dienen, Oberflächen
möglichst realistisch und physikalisch korrekt darzustellen. Eine
Verallgemeinerung der BRDF auf Texturen stellt die BTF (Bidirectional
Texturing Function)
dar.''~\parencite{wikipedia_the_free_encyclopedia_bidirektionale_2014}

\citeauthor{burley_physicall-based_2012} liefert
mit~\citetitle{burley_physicall-based_2012} eine gute Übersicht,
wie solche Verfahren umgesetzt werden
können~\parencite{burley_physicall-based_2012}.
\citeauthor{bagher_accurate_2012} bieten
in~\citetitle{bagher_accurate_2012} eine Fülle an Materialien und
Eigenschaften~\parencite{bagher_accurate_2012}.

\subsection{Optimierungsverfahren}
\label{subsec:further_work:optimisation}

Zur Beschleunigung des Sphere Tracing Verfahrens existieren diverse
Möglichkeiten. So stellen z.B.~\citeauthor{keinert_enhanced_2014}
in~\citetitle{keinert_enhanced_2014} diverse Methoden zur
Optimierung und Beschleunigung des Renderings sowie Methoden zum Finden
von Schnittpunkten vor~\parencite{keinert_enhanced_2014}.~\citeauthor{seven_rendering_2012} stellt
in~\citetitle{seven_rendering_2012} eine Methode vor, wie Sphere Tracing
mittels Aussenden von Kegelförmigen (Primär-) Strahlen optimiert werden
kann~\parencite{seven_rendering_2012}.

