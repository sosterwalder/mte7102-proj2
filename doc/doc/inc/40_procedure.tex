% -*- coding: UTF-8 -*-
% vim: autoindent expandtab tabstop=4 sw=4 sts=4 filetype=tex
% vim: spelllang=de spell
% chktex-file 27 - disable warning about missing include files

\chapter{Vorgehen}
\label{chap:procedure}

\section{Arbeitsorganisation}
\label{sec:organization}

\subsection{Treffen}
\label{subsec:meetings}

Besprechungen mit dem Betreuer der Arbeit halfen, die gesteckten Ziele zu
erreichen und Fehlentwicklungen zu vermeiden. Der Betreuer unterstützte den
Autor dabei mit Vorschlägen. Insgesamt fanden vier Treffen statt. Sie wurden in
Form eines Protokolles festgehalten. Das Protokoll findet sich
unter~\autoref{chap:10_meeting_minutes}.

\section{Projektphasen}
\label{sec:project_schedule}

\subsection{Meilensteine}
\label{subsec:milestones}

Um bei der Arbeit ein möglichst strukturiertes Vorgehen einzuhalten,
wurden folgende Projektphasen gewählt:

\begin{itemize}
    \item Start der Projektarbeit
    \item Erarbeitung und Festhalten der Anforderungen
    \item Erstellung eines~\hyperref[chap:prototype]{Prototypen}
    \item Erstellung der abschliessenden Dokumentation aufgrund der gewonnen
        Erkenntnisse
\end{itemize}

Die Phasen \textit{Erstellung eines Prototypen} sowie~\textit{Erstellung der
    abschliessenden Dokumentation} liefen parallel ab. Erkenntnisse einer Phase
flossen jeweils in die andere Phase ein.

\subsection{Zeitplan / Projektphasen}
\label{subsec:timeschedule}

\begin{figure}[H]
    \begin{ganttchart}[
        vgrid,
        x unit=0.43cm,
        bar/.append style={fill=bfhgrey!50},
    ]{1}{25}
        \gantttitle{2016}{25} \ganttnewline{}
        \gantttitlelist{8,...,32}{1} \ganttnewline{} % chktex 11: Disable "you should use \ldots to achieve.."
        \ganttbar{Projektstart}{1}{1} \ganttnewline{}
        \ganttlinkedbar{Anforderungen}{2}{3} \ganttnewline{}
        \ganttlinkedbar{Erstellung Prototyp}{4}{10} \ganttnewline{}
        \ganttlinkedbar{Dokumentation}{14}{23} \ganttnewline{}
        \ganttlinkedbar{Korrekturen}{23}{24} \ganttnewline{}
        \ganttlinkedmilestone{Abgabe Dokumentation}{24} \ganttnewline{}
        \ganttbar{Erstellung Prototyp}{23}{23} \ganttnewline{}
        \ganttbar{Vorbereitung Präsentation/Verteidigung}{24}{25} \ganttnewline{}
        \ganttmilestone{Präsentation/Verteidigung}{25}
    \end{ganttchart}
    \caption{Zeitplan; Der Titel stellt Jahreszahlen, der Untertitel
    Kalenderwochen dar}\label{fig:timeschedule}
\end{figure}

\subsubsection{Anforderungen}
\label{ssubsec:requirements}

In dieser Phase wurde das Ziel dieser Projektarbeit festgelegt. Ausgehend vom
Ziel wurden die dazu erforderlichen Projektphasen festgelegt. Bei diesen
handelt es sich um die hier beschriebenen Projektphasen.

\subsubsection{Erstellung Prototyp}
\label{ssubsec:prototype}

Ausgehend von den Anforderungen und vor allem der Vision, wurde ein Prototyp
erstellt um die Machbarkeit aufzuzeigen und um Erkenntnisse zu gewinnen.

\subsubsection{Dokumentation}
\label{ssubsec:documentation}

Die vorliegende Arbeit entspricht der Dokumentation. Sie sollte während der
gesamten Projektarbeit stetig erweitert werden. Allerdings wurde zu viel Fokus
auf den Prototypen gelegt und somit wurde die Dokumentation eher erst spät
richtig angegangen. Sie diente zur Reflexion von fertiggestellten Teilen.

% \section{Technologien}
% \label{sec:technologies}
% 
% \subsection{Tools und Software}
% \label{subsec:tools_software}
% 
% \subsubsection{Dokumentation und Prototyp}
% \label{ssubsec:tools_software:documentation_prototype}
% 
% \begin{description}
%     \item[\LaTeX] Eine Makro-Sammlung für das \TeX-System. Wurde zur
%         Erstellung dieser Dokumentation eingesetzt. Diese Dokumentation
%         wurde mittels \LaTeX{} geschrieben.
%     \item[Make] Build-Automations-Werkzeug, wurde zur Erstellung dieses Dokumentes eingesetzt.
%     \item[CMake] Build-Automations-Werkzeug, wurde zur Erstellung des
%         \hyperref[chap:prototype]{Prototypen} eingesetzt.
%     \item[zotero] Ein freies, quelloffenes Literaturverwaltungsprogramm
%         zum Sammeln, Verwalten und Zitieren unterschiedlicher Online-
%         und Offline-Quellen~\parencite{wikipedia_foundation_zotero_2015}.
%     \item[VIM] Vi IMproved. Ein freier, quelloffener Texteditor zur
%         Textbearbeitung. Er wurde zum Verfassen der Dokumentation sowie zur
%         Entwicklung des~\hyperref[chap:prototype]{Prototypen} eingesetzt.
%     \item[LLVM] Low Level Virtual Machine. Eine Compiler-Architektur zum
%         Kompilieren von Applikationen. Wurde zur Kompilation des
%         \hyperref[chap:prototype]{Prototypen} eingesetzt.
%     \item[clang] Ein C-Sprachen-Frontend für LLVM.\ Wurde zur Kompilation
%         des~\hyperref[chap:prototype]{Prototypen} eingesetzt.
%     \item[Papyrus] Ein quelloffenes Werkzeug zur Erstellung von UML- und
%         SysML-Modellen~\parencite{the_eclipse_foundation_papyrus_2015}.
%     \item[Pencil] Ein quelloffenes Werkzeug zur Erstellung von Prototypen von
%         grafischen Benutzeroberflächen~\parencite{evolus_pencil_2015}.
% \end{description}
% 
% 
% \subsubsection{Arbeitsorganisation}
% \label{ssubsec:tools_software:job_organisation}
% 
% \begin{description}
%     \item[Git] Freie Software zur verteilten Versionsverwaltung, wurde
%         für die Entwicklung dieser Dokumentation sowie
%         des~\hyperref[chap:prototype]{Prototypen} verwendet. Die
%         Projektarbeit findet sich
%         unter~\href{https://www.github.com/sosterwalder/mte7101-project1}{GitHub}\footnote{
%         \href{https://www.github.com/sosterwalder/mte7101-project1}{https://www.github.com/sosterwalder/mte7101-project1}
%     }.
%     \item[GitHub] Eine freie Hosting-Platform für Git mit Weboberfläche.
% \end{description}

\section{Standards und Richtlinien}
\label{sec:standards_guidelines}

\subsection{Programmcode}
\label{subsec:standards_guidelines:code}

Der Programmcode des~\hyperref[chap:prototype]{Prototypen}, welcher in C++ geschrieben wurde, folgt
den offiziellen Richtlinien für C++ von Google~\footnote{
    \href{https://google.github.io/styleguide/cppguide.html}{
        https://google.github.io/styleguide/cppguide.html
    }
}.

\subsection{Diagramme}
\label{subsec:standards_guidelines:diagrams}

Alle Diagramme folgen dem UML 2 Standard~\footnote{
    \href{http://www.omg.org/spec/UML/}{http://www.omg.org/spec/UML/}}.

\subsection{Projekt-Struktur}
\label{subsec:standards_guidelines:project_structure}

Um die Übersicht zu wahren und den Verwaltungsaufwand minimal zu halten,
wurde eine entsprechende Projekt-Struktur gewählt. Diese ist
in~\autoref{lst:project_structure} ersichtlich.

\begin{listing}
    \VerbatimInput[label=Projekt-Struktur,frame=single,numbers=left,firstline=19,lastline=39]{../../README.md}
    \caption{Projekt-Struktur.}\label{lst:project_structure}
\end{listing}
