% -*- coding: UTF-8 -*-
% vim: autoindent expandtab tabstop=4 sw=4 sts=4 filetype=tex
% vim: spelllang=de spell
% chktex-file 27 - disable warning about missing include files

\chapter{Rendering}
\label{chap:rendering}


Um schliesslich die erstellte Echtzeit-Animationen auch darstellen zu können
--- im Editor und im Player --- wird ein Rendering-Verfahren benötigt.

Als Rendering-Verfahren soll primär das Ray-Tracing-Verfahren Sphere-Tracing
zum Einsatz kommen. Auf dieses wird hier nicht eingegangen, da dieses
bereits ausführlich in der vorhergehenden Projektarbeit behandelt wurde
(siehe~\cite{osterwalder_sven_volume_2016}).

Da bei der vorhergehenden Projektarbeit die~\textit{Berechnung von
    Normalen-Vektoren}, die \textit{Repetitions-Operation} und deren
\textit{Komplexität} nicht ausreichend geklärt wurden beziehungsweise bei der
Projektarbeit Fragen dazu aufkamen, werden die Punkte an dieser Stelle genauer
erläutert. Weiter wird ein möglicher Ansatz gezeigt, um \textit{herkömmliche
    3D-Modelle (Meshes)} mittels Sphere-Tracing darstellen zu können.

Als Grundlage für das Rendering wird OpenGL gewählt, da diese Bibliothek
plattformübergreifend und quelloffen ist. Mit der Einführung der Version 3 von
OpenGL, beziehungsweise Version 2 von OpenGL ES, wurde die Rendering-Pipeline
dahingehend geändert, dass diese nun nicht mehr fix vorgegeben sondern frei
programmierbar
ist\cite{opengl_foundation_fixed_2015}~\cite{opengl_foundation_rendering_2015}.
So gesehen findet das Rendering spätestens seit OpenGL Version 3
beziehungsweise OpenGL ES Version 2 immer per Shader statt. Ist der Aufwand der 
Programmierung der Rendering-Pipeline initial grösser, so ist eine frei
programmierbare Pipeline jedoch viel flexibler.

Genau dieser Ansatz soll auch in der hier vorgestellten Software-Architektur
für das Rendering verwendet werden: Ein modulares Rendering, welches jederzeit
ersetzt werden kann. Da alles via Shader gerendert wird, ist dies so gesehen
bereits gegeben. Szenen können --- im Falle von Sphere-Tracing --- rein aus
internen Funktionen und Daten bestehen oder aber --- im Falle der Verwendung
von herkömmlichen Modellen (Meshes) --- externe Daten beinhalten. So wäre
beispielsweise zusätzlich zu Sphere-Tracing
Deferred-Rendering~\footnotetext{\url{https://sites.google.com/site/richgel99/home}}~\parencite{saito_comprehensible_1990}
denkbar. Die Schnittstelle zwischen der Applikation und Shader, in Form von
Uniform-Variablen, wird aber so oder benötigt um auf der CPU berechnete
Matrizen an den Shader respektive die GPU weiterzureichen.

% -*- coding: UTF-8 -*-
% vim: autoindent expandtab tabstop=4 sw=4 sts=4 filetype=tex
% vim: spelllang=de spell
% chktex-file 27 - disable warning about missing include files

\section{Berechnung von Normalen-Vektoren}
\label{sec:rendering:normals}

Folgender Abschnitt basiert auf~\cite[S. 42 bis
43]{osterwalder_sven_volume_2016}.

Die Oberflächen-Normale kann gemäss~\citeauthor{hart_ray_1989} mittels dem
Gradienten des Distanzfeldes für einen bestimmten Punkt auf einer impliziten
Oberfläche berechnet werden~\parencite[S. 292 bis 293]{hart_ray_1989}.

Verwendet man dies nun zusammen mit den entsprechenden Distanzfunktionen, so
ergeben sich folgende Gleichungen:

\begin{gather}
    \bm{n}_{x} = f(x + \varepsilon, y, z) - f(x - \varepsilon, y, z) \\
    \bm{n}_{y} = f(x, y + \varepsilon,  z) - f(x, y - \varepsilon,  z) \\
    \bm{n}_{z} = f(x, y, z + \varepsilon) - f(x, y, z - \varepsilon) \\
\end{gather}

Dabei ist $\bm{n} = \begin{bmatrix} x_{n} \\ y_{n} \\ z_{n} \end{bmatrix}$ die
Normale der Oberfläche in Form eines Vektors mit drei Komponenten und $f$ eine
Distanzfunktion~\parencites[S. 292 bis 293]{hart_ray_1989}[S.
13]{hart_ray_1993}.

Es werden also insgesamt sechs Punkte abgetastet. $\varepsilon$ ist dabei ein
Wert um alle benachbarten Punkte eines gegebenen Punktes der impliziten
Oberfläche entlang der Koordinatenachsen zu erhalten. Da die Berechnung der
Normalen somit direkt von $\varepsilon$ abhängt, wird für $\varepsilon$
üblicherweise ein möglichst kleiner Wert verwendet.  \citeauthor{hart_ray_1989}
geben $\varepsilon$ als die minimale Inkrementation eines (Licht-) Strahles an.
Die Normale der Oberfläche sollte schliesslich noch normalisiert werden.

``Liefert die oben genannte Gradiente, bestehend aus 6 Punkten, eine zu
geringe Genauigkeit, so kann diese gemäss \citeauthor{hart_ray_1989}
erweitert werden~\parencite[S. 293]{hart_ray_1989}.\\
Die Erweiterung erfolgt durch Hinzunahme von Punkten, welche eine
gemeinsame Kante haben. Dies erzeugt eine Gradiente bestehend aus 18
Punkten. Werden noch die Punkte hinzugenommen, welche gemeinsame
Eckpunkte haben, so ergibt sich eine Gradiente bestehend aus 26
Punkten~\parencite[S. 293]{hart_ray_1989}.''~\cite[S.
43]{osterwalder_sven_volume_2016}


% -*- coding: UTF-8 -*-
% vim: autoindent expandtab tabstop=4 sw=4 sts=4 filetype=tex
% vim: spelllang=de spell
% chktex-file 27 - disable warning about missing include files

\section{Repetitions-Operation}
\label{sec:rendering:modulo}

Wie unter~\cite[S. 37ff]{osterwalder_sven_volume_2016} aufgeführt, existieren
für implizite Oberflächen diverse Operationen, wie Distanz- und
Domänen-Operationen sowie -Deformationen.

Möchte man nun aber dasselbe Objekt mehrmals darstellen, so repetiert man
dieses anhand einer oder mehrerer Achsen. Dies geschieht durch Anpassung der
Domäne, also der Anpassung des Raumes, in dem sich eine implizite Oberfläche
befindet. Es handelt sich dabei also um eine Domänen-Operation.

Die Anpassung der Domäne kann mittels der Modulo-Operation (\textit{mod}, \%)
vorgenommen werden. Die Modulo-Operation gibt den (vorzeichenbehafteten) Rest
einer Division zurück~\cite{maignan_integer_2008}.

\begin{gather}
    d(\bm{x}, \bm{c}) = \bm{x} \mod \bm{c} - {\bm{c} \over 2}\\
    \text{repeat}(\bm{x}, \bm{c}) = f(d(\bm{x}, \bm{c}))
\end{gather}

Dabei ist $\bm{x}$ der Punkt einer impliziten Oberfläche $f$ und $\bm{c}$ der
gewünschte Abstand zwischen den Objekten.

Das Objekt wird erst mit der Hälfte des Abstandes $c$, also $c \over 2$, anhand
der Koordinatenachsen verschoben, um es im Rahmen des gewünschten Abstandes
$c$ zu zentrieren.  Danach wird die Repetition der Domäne angewendet. Danach
wird das Objekt zum Ursprung seines Koordinatensystemes zurück verschoben.

% -*- coding: UTF-8 -*-
% vim: autoindent expandtab tabstop=4 sw=4 sts=4 filetype=tex
% vim: spelllang=de spell
% chktex-file 27 - disable warning about missing include files

\subsection{Komplexität}
\label{subsec:rendering:modulo:complexity}

Es stellt sich nun die Frage, wie hoch die zusätzliche Komplexität bei
Anwendung der Modulo-Operation ist. Der Autor geht davon aus, dass der Aufwand
zur Darstellung von sich wiederholenden Objekten linear ist. Die Methode zum
Abtasten der Distanz, \textit{castRay}, ändert sich in diesem Sinne linear mit
der Anzahl Objekte, die dargestellt werden sollen. Je mehr Objekte, desto höher
der Aufwand. Nimmt man nun an, dass ein Objekt, welches wiederum eine
Komposition von mehreren Objekten sein kann, wiederholt wird, so wird dennoch
nach Erreichen der maximalen Anzahl Schritte oder der maximalen Distanz
abgebrochen, was linear ist. Vereinfacht gesagt wird schliesslich pro Iteration
nur zurückgegeben, ob auf ein Objekt getroffen wurde oder nicht. Die Berechnung
der Beleuchtung respektiv der Farbe einer Oberfläche erhöht die Komplexität
natürlich nochmals.  Die Thematik soll an dieser Stelle jedoch nicht weiter
vertieft werden, da dies den Rahmen dieser Projektarbeit deutlich sprengen
würde.

% -*- coding: UTF-8 -*-
% vim: autoindent expandtab tabstop=4 sw=4 sts=4 filetype=tex
% vim: spelllang=de spell
% chktex-file 27 - disable warning about missing include files

\section{Meshes}
\label{sec:rendering:meshes}

Während der Präsentation der vorhergehenden Projektarbeit, MTE7101, kam die
Frage auf, ob es auch möglich ist ``konventionelle'' 3D-Modelle (Meshes) mittels
Sphere-Tracing darzustellen. Die Modellierung bei der Anwendung von Sphere-Tracing findet 
üblicherweise mittels Distanzfunktionen statt und ist daher rein Mathematisch.
Sphere-Tracing nutzt Distanzfunktionen und Distanzfelder zur Darstellung von
impliziten Oberflächen~\cite[S. 31]{osterwalder_sven_volume_2016}.

Betrachtet man nun aber Meshes genauer, so handelt es sich prinzipiell auch um
Tiefeninformationen --- wie bei einem Distanzfeld. Somit müsste es möglich sein
ein Distanzfeld aus einem Mesh zu erzeugen und in einer dreidimensionalen
Textur abzuspeichern. Dazu könnte ein dreidimensionales Raster erzeugt und das
Mesh darin abgelegt werden. Danach müsste man die Distanzwerte von jedem Punkt
des Rasters zur Oberfläche des Meshes in der dreidimensionalen Textur ablegen.
Das Distanzfeld wäre dann zwar initial nicht $\in \mathbb{R}^{3}$, wie für das
Sphere-Tracing benötigt, das Speichern in eine dreidimensionale Textur würde
die Werte jedoch dann wieder auf [0, 1] beschränken.

Dem Autor dieser Arbeit ist bisher kein solches Verfahren bekannt. Ob die oben
beschriebene Methode wirklich funktioniert und wie gut die Resultate wären
müsste ausprobiert werden.


