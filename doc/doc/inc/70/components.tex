% -*- coding: UTF-8 -*-
% vim: autoindent expandtab tabstop=4 sw=4 sts=4 filetype=tex
% vim: spelllang=de spell
% chktex-file 27 - disable warning about missing include files

Details der einzelnen Komponenten finden sich in der nachfolgenden Tabelle.

% Sigh..
% FYI:  Tabular and a fixed text-width of 79 sucks!
% FYI2: set cursorcolumn with a lot of text sucks too!

\begin{tabular}{llp{10cm}l}
    \toprule
    \textbf{Komponente} & \textbf{Version} & \textbf{Beschreibung} & \textbf{Verweise} \\
    \midrule
    C++        & 11      & Objektorientierte Programmiersprache
                           &\protect\footnotemark\\

    OpenGL     & 4.5     & Plattformunabhängige Programmierschnittstelle zur
                           Entwicklung von 2D- und 3D-Computergrafikanwendungen
                           \parencite{wikipedia_the_free_encyclopedia_opengl_2015}.
                           &\protect\footnotemark\\

    GLFW       & 3.1.2   & OpenGL-Bibliothek, welche die Erstellung und
                           Verwaltung von Fenstern sowie OpenGL-Kontexte
                           vereinfacht
                           \parencite{wikipedia_the_free_encyclopedia_glfw_2015}.
                           &\protect\footnotemark\\

    GLEW       & 1.13    & OpenGL Extension Wrangler. Bibliothek zum Abfragen
                           und Laden von OpenGL-Erweiterungen (Extensions)
                           \parencite{wikipedia_the_free_encyclopedia_opengl_2015-1}.
                           &\protect\footnotemark\\

    CMake      & 3.3.2   & Software zur Verwaltung von Build-Prozessen von 
                           Software
                           &\protect\footnotemark\\

    GCC        & 5.2     & GNU Compiler Collection. Compiler-System des
                           GNU-Projektes
                           &\protect\footnotemark\\

    Boost      & 1.59.0  & Freie Bibliothek bestehend aus einer Vielzahl von
                           Bibliotheken, die den unterschiedlichsten
                           Aufgaben von Algorithmen auf Graphen über 
                           Metaprogrammierung bis hin zu Speicherverwaltung
                           dienen
                           \parencite{wikipedia_the_free_encyclopedia_boost_2015}.
                           &\protect\footnotemark\\
    \bottomrule
\end{tabular}
\footnotetext{\url{http://www.iso.org/iso/iso_catalogue/catalogue_tc/catalogue_detail.htm?csnumber=50372}}
\footnotetext{\url{https://www.opengl.org/registry/doc/glspec45.core.pdf}}
\footnotetext{\url{http://www.glfw.org}}
\footnotetext{\url{http://glew.sourceforge.net}}
\footnotetext{\url{https://www.cmake.org}}
\footnotetext{\url{http://gcc.gnu.org}}
\footnotetext{\url{http://www.boost.org}}
