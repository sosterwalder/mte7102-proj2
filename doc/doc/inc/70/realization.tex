% -*- coding: UTF-8 -*-
% vim: autoindent expandtab tabstop=4 sw=4 sts=4 filetype=tex
% vim: spelllang=de spell
% chktex-file 27 - disable warning about missing include files

\section{Umsetzung}
\label{sec:realization}

Nachfolgend werden die
gemäss~\autoref{sec:description_implicit_surfaces} umgesetzten Konzepte
genauer beschrieben. Es handelt sich dabei um Ausschnitte des
umgesetzten Fragment-Shaders.

Das eigentliche Sphere Tracing geschieht in der Funktion
\textit{castRay}.  Diese hat als Parameter den Ursprung eines Strahles
(\textit{vec3 rayOrigin}), die Richtung eines Strahles (\textit{vec3
    rayDirection}), die maximale Distanz (\textit{float maxDistance}),
welche berechnet werden soll, die Präzision (\textit{float precision})
sowie die Anzahl Durchgänge (\textit{int steps}).

Ein Vergleich der letzten drei Parameter --- die maximale Distanz, die
Präzision sowie die Anzahl Durchgänge --- findet sich in den
Tabellen~\ref{table:sphere_tracing_distance},~\ref{table:sphere_tracing_precision}
und~\ref{table:sphere_tracing_steps}.

Maschinell bedingt kommen bei den Berechnungen Fliesskommazahlen zum
Einsatz, da die Resultate möglichst genau sein sollen. Aufgrund der
endlichen Mantisse bei der Darstellung von Fliesskommazahlen lassen sich
diese auf einem Computer jedoch nicht beliebig genau darstellen und
damit berechnen. Rundung und damit verbundene Rundungsfehler sind
unvermeidlich.  Um dennoch eine gute Annäherung an Null-Werte zu
erhalten, wird ein Parameter ($\varepsilon$) zur Steuerung der minimalen
Distanz zu einer Oberfläche verwendet. Dieser kommt bei den
Funktionen~\textit{render}, \textit{castRay} sowie \textit{calcShadows}
in Form der Variablen \textit{minimalDistance} bzw.  \textit{precision}
zum Einsatz.

\begin{minipage}{\linewidth}
\begin{lstlisting}[language=GLSL,caption={Umsetzung des Sphere Tracings in
        GLSL.},label={alg:glsl_sphere_tracing},captionpos=b,emph={castRay}]
// Casts a ray from given origin i given direction. Stops at given
// maximal distance and after given amount of steps. Maintains given
// precision.
vec2 castRay(in vec3 rayOrigin, in vec3 rayDirection, in float maxDistance, in float precision, in int steps)
{
    float latest    = precision * 2.0;
    float distance  = 0.0;
    float type      = -1.0;
    vec2  res       = vec2(-1.0, -1.0);

    for(int i = 0; i < steps; i++) {
        if (abs(latest) < precision || distance > maxDistance) {
            continue;
        }

        vec2 result = scene(rayOrigin + rayDirection * distance);

        latest     = result.x;
        type       = result.y;
        distance  += latest;
    }

    if (distance < maxDistance) {
        res = vec2(distance, type);
    }

    return res;
}
\end{lstlisting}
\end{minipage}

% \begin{table}[H]
%     \centering
%     \caption{Vergleich des Distanz-Parameters anhand einer Beispielszene.}\label{table:sphere_tracing_distance}
%     \begin{tabular}{p{0.3\textwidth}p{0.3\textwidth}p{0.3\textwidth}}
%         \toprule
%             \textbf{Distanz: \textit{100.0}} &
%             \textbf{Distanz: \textit{9.0}}   &
%             \textbf{Distanz: \textit{6.3}}   \\
%         \cmidrule(r){1-1}\cmidrule(lr){2-2}\cmidrule(l){3-3}
%             \includegraphics[width=0.3\textwidth]{img/sphere_tracing_distance_full.pdf}
%             \newline
%             Alle in der Szene definierten Objekte sind sichtbar. &
%             \includegraphics[width=0.3\textwidth]{img/sphere_tracing_distance_less.pdf} \newline
%             Es ist nur noch das zum Betrachter näher liegende Objekt sichtbar. &
%             \includegraphics[width=0.3\textwidth]{img/sphere_tracing_distance_min.pdf} \newline
%             Es ist nur noch das zum Betrachter näher liegende Objekt
%             sichtbar, jedoch nicht mehr vollständig. \\
%         \bottomrule
%     \end{tabular}
% \end{table}
% 
% \begin{table}[H]
%     \centering
%     \caption{Vergleich des Präzisions-Parameters anhand einer Beispielszene.}\label{table:sphere_tracing_precision}
%     \begin{tabular}{p{0.3\textwidth}p{0.3\textwidth}p{0.3\textwidth}}
%         \toprule
%             \textbf{Präzision: \textit{0.00001}} &
%             \textbf{Präzision: \textit{0.1}}     &
%             \textbf{Präzision: \textit{1.0}}     \\
%         \cmidrule(r){1-1}\cmidrule(lr){2-2}\cmidrule(l){3-3}
%             \includegraphics[width=0.3\textwidth]{img/sphere_tracing_precision_full.pdf} \newline
%             Die Kugel weist keinerlei sichtbare Abstufungen auf. &
%             \includegraphics[width=0.3\textwidth]{img/sphere_tracing_precision_min.pdf} \newline
%             Die Kugel weist am unteren linken Rand sichtbare Farbabstufungen auf. &
%             \includegraphics[width=0.3\textwidth]{img/sphere_tracing_precision_pos.pdf} \newline
%             Die Kugel wird nicht mehr korrekt dargestellt. \\
%         \bottomrule
%     \end{tabular}
% \end{table}
% 
% \begin{table}[H]
%     \centering
%     \caption{Vergleich des Parameters zur Bestimmung der Anzahl der
%         Durchgänge anhand einer Beispielszene.}\label{table:sphere_tracing_steps}
%     \begin{tabular}{p{0.3\textwidth}p{0.3\textwidth}p{0.3\textwidth}}
%         \toprule
%             \textbf{Anzahl Schritte: \textit{100}} &
%             \textbf{Anzahl Schritte: \textit{10}}  &
%             \textbf{Anzahl Schritte: \textit{5}}   \\
%         \cmidrule(r){1-1}\cmidrule(lr){2-2}\cmidrule(l){3-3}
%             \includegraphics[width=0.3\textwidth]{img/sphere_tracing_steps_full.pdf} \newline
%             Alle in der Szene definierten Objekte sind korrekt sichtbar. &
%             \includegraphics[width=0.3\textwidth]{img/sphere_tracing_steps_less.pdf} \newline
%             Die Grenze zwischen den in der Szene definierten Objekten
%             ist bereits nicht mehr eindeutig. &
%             \includegraphics[width=0.3\textwidth]{img/sphere_tracing_steps_min.pdf} \newline
%             Die Szene ist als solche nicht mehr erkennbar, es erfolgt
%             keine klare Trennung zwischen den einzelnen Objekten.  \\
%         \bottomrule
%     \end{tabular}
% \end{table}

Von den unter~\autoref{subsec:implicit_surfaces_ops} beschriebenen
Operationen wurden die Operationen \textit{Vereinigung}, \textit{Subtraktion} sowie
\textit{Intersektion} umgesetzt.

\begin{minipage}{\linewidth}
\begin{lstlisting}[language=GLSL,caption={Umsetzung der Operationen
        \textit{Vereinigung}, \textit{Subtraktion} sowie
        \textit{Intersektion} für implizite Oberflächen in
        GLSL.},label={alg:glsl_ops},captionpos=b,emph={subtract,merge,intersect}]
// Returns the signed distance for a substraction of given signed
// distance a to signed distance b.
float subtract(float a, float b)
{
    return max(-b, a);
}

// Returns the signed distance for a merge of given signed
// distance a and signed distance b.
vec2 merge(vec2 a, vec2 b)
{
    return min(a, b);
}

// Returns the signed distance for a intersection  of given signed
// distance a and signed distance b.
float intersect(float a, float b)
{
    return (a > b) ? a : b;
}
\end{lstlisting}
\end{minipage}

Von den unter~\autoref{subsec:implicit_surfaces_primitives} beschriebenen
Primitiven wurden die Primitiven \textit{Ebene} sowie \textit{Kugel} umgesetzt.

\begin{minipage}{\linewidth}
\begin{lstlisting}[language=GLSL,caption={Umsetzung der Primitiven
        \textit{Ebene} und \textit{Kugel} in Form von impliziten
        Oberflächen in
        GLSL.},label={alg:glsl_primitives},captionpos=b,emph={plane,sphere,box}]
// Returns the signed distance to a plane for the given position.
float plane(vec3 position)
{
    return position.y;
}

// Returns the signed distance to a sphere with given radius for the
// given position.
float sphere(vec3 position, float radius)
{
    return length(position) - radius;
}

// Returns the signed distance to a box with given dimension for the
// given position.
float box(vec3 position, vec3 dimension)
{
    position = abs(position) - dimension;
    return max(max(position.x, position.y), position.z);
}
\end{lstlisting}
\end{minipage}

Verwendete Parameter sind jeweils die (gewünschte) Position des Objektes
(\textit{vec3 position}) sowie der Radius bzw.\ die Dimension
(\textit{float radius} bzw. \textit{vec3 dimension}).

Als Beleuchtungsmodell wurde das
unter~\autoref{sec:rendering_implicit_surfaces} beschriebene
Phong-Beleuchtungsmodell umgesetzt.

\begin{minipage}{\linewidth}
\begin{lstlisting}[language=GLSL,caption={Umsetzung des
        Phong-Beleuchtungsmodelles in
        GLSL.},label={alg:glsl_lighting},captionpos=b,emph={calcLighting}]
// Calculates the lighting for the given position, normal and direction,
// the given light (position and color) respecting the 'material'.
//
// This is mainly applying the phong lighting model inlcuding shadows.
//
// Returns the calculated color as three-dimensional vector.
vec3 calcLighting(vec3 normal, vec3 rayDirection) {

    vec3 lightDirection     = normalize(vec3(0.0, 4.0, 5.0));

     float kDirectLight      = 0.1;
     float shadows           = calcShadows(position, lightDirection);
     vec3  direct            = vec3(kDirectLight * shadows);

    vec3 ambientColor       = vec3(0.05, 0.15, 0.2);
    float kAmbient          = clamp(0.5 + 0.5 * normal.y, 0.0, 1.0);
    vec3 ambient            = kAmbient * ambientColor;

    vec3 diffuseColor       = vec3(0.2, 0.6, 0.8);
    float kDiffuse          = clamp(dot(lightDirection, normal), 0.0, 1.0);
    vec3 diffuse            = kDiffuse * diffuseColor;

    vec3 specularColor      = vec3(1.0);
    float kSpecularExponent = 24.0;
    vec3 h                  = normalize(-rayDirection + lightDirection);
    float nFacing           = clamp(dot(lightDirection, normal), 0.0, 1.0);
    float kSpecular         = pow(clamp(dot(h, normal), 0.0, 1.0), kSpecularExponent);
    vec3 specular           = nFacing * kSpecular * specularColor;

     vec3 light              = ambient + diffuse + specular + direct;
     vec3 color              = material * light;
 
     return color;
}
\end{lstlisting}
\end{minipage}

Als Parameter werden hier ein Normalvektor einer Oberfläche ($\bm{n}$)
sowie die Richtung eines eingehenden Strahles (also die Blickrichtung
des Betrachters bzw.\ der Kamera, $\vv{V}$) benötigt.
Wie oben ersichtlich, wird zuerst die Richtung der Lichtquelle definiert,
danach werden die einzelnen Anteile des Lichtes $I_{\text{ambient}}$,
$I_{\text{diffuse}}$ sowie $I_{\text{sepcular}}$ berechnet.

Die Normale einer Oberfläche wird wie
unter~\autoref{sec:rendering_implicit_surfaces_lighting} beschrieben
berechnet.

\begin{minipage}{\linewidth}
\begin{lstlisting}[language=GLSL,caption={Berechnung der Normalen einer
        impliziten Oberfläche in
        GLSL.},label={alg:glsl_normal},captionpos=b,emph={calcNormal}]
// Calculates the normal vector for given position with respect to a
// certain offset given by epsilon.
vec3 calcNormal(in vec3 position, in float epsilon) {

    vec3 eps = vec3(epsilon, 0.0, 0.0);
    vec3 normal  = vec3(
        scene(position + eps.xyy).x - scene(position - eps.xyy).x,
        scene(position + eps.yxy).x - scene(position - eps.yxy).x,
        scene(position + eps.yyx).x - scene(position - eps.yyx).x
    );

    return normalize(normal);
}
}
\end{lstlisting}
\end{minipage}

Verwendete Parameter sind ein Punkt der Oberfläche eines Objektes
(\textit{vec3 position}) sowie die minimale Inkrementation eines
(Licht-) Strahles ($\varepsilon$). Wie zuvor erwähnt, sollte
für $\varepsilon$ ein möglichst kleiner Wert gewählt werden. Daher wird
standardmässig der Wert \textit{0.1} eingesetzt.

Bei der Funktion \textit{scene} handelt es sich um eine Distanzfunktion
$f$ gemäss~\autoref{ssubsec:distance_functions}. Diese definiert
schliesslich, was an einem gegebenen Punkt dargestellt wird.

\begin{minipage}{\linewidth}
\begin{lstlisting}[language=GLSL,caption={Distanzfunktion $f$ in
        GLSL.},label={alg:glsl_distance_func},captionpos=b,emph={scene}]
// Defines the scene which will be drawn at given position.
float scene(in vec3 position) {
{
    float sphereRadius = 1.0;
    vec3  sphereOffset = vec3(-2.0, 1.0, -2.0);

    float res = sphere(position - sphereOffset, sphereRadius);

    return res;
}
\end{lstlisting}
\end{minipage}

In diesem Beispiel wird eine Kugel mit Radius \textit{1.0} dargestellt.
Diese wird auf der \textit{X}-Achse um -2 Einheiten, auf der
\textit{Y}-Achse um 1 Einheit und auf der \textit{Z}-Achse um -2
Einheiten verschoben (Translation).

Wie in~\autoref{sec:rendering_implicit_surfaces_shadows} beschrieben,
wurden weiche Schatten implementiert.

\begin{minipage}{\linewidth}
\begin{lstlisting}[language=GLSL,caption={Funktion zur Berechnung von
        weichen Schatten  in
        GLSL.},label={alg:glsl_soft_shadows},captionpos=b,emph={calcShadows}]
// Calcuates soft shadows for the given ray origin and direction.
//
// Returns a shadow color for the given origin between 0.0 and 1.0.
float calcShadows(in vec3 rayOrigin, in vec3 rayDirection)
{
    float shadow            = 1.0;
    float minimalDistance   = 0.01;
    float maximalDistance   = 2.5;
    float convergePrecision = 0.000001;
    float kShadow           = 8.0;
    float currentDistance   = minimalDistance;

    while (currentDistance < maximalDistance) {
        vec3 ray = rayOrigin + rayDirection * currentDistance;
        float estimatedDistance = scene(ray);

        if (estimatedDistance < convergePrecision) {
            return 0.0;
        }

        float penumbraFactor = estimatedDistance / currentDistance;
        shadow = min(shadow, kShadow * penumbraFactor);
        currentDistance += estimatedDistance;
    }

    return clamp(shadow, 0.0, 1.0);

}
\end{lstlisting}
\end{minipage}

Verwendete Parameter sind der Ursprung (\textit{vec3 rayOrigin}) sowie
die Richtung (\textit{vec3 rayDirection}) eines Strahles.

% \begin{table}[H]
%     \centering
%     \caption{Vergleich des Skalierungsfaktors \textit{kShadow} für
%         weiche Schatten anhand einer
%         Beispielszene.}\label{table:sphere_tracing_shadows}
%     \begin{tabular}{p{0.3\textwidth}p{0.3\textwidth}p{0.3\textwidth}}
%         \toprule
%             \textbf{kShadow: \textit{8.0}} &
%             \textbf{kShadow: \textit{16.0}}   &
%             \textbf{kShadow: \textit{32.0}}   \\
%         \cmidrule(r){1-1}\cmidrule(lr){2-2}\cmidrule(l){3-3}
%             \includegraphics[width=0.3\textwidth]{img/sphere_tracing_shadows_8.pdf} \newline &
%             \includegraphics[width=0.3\textwidth]{img/sphere_tracing_shadows_16.pdf} \newline &
%             \includegraphics[width=0.3\textwidth]{img/sphere_tracing_shadows_32.pdf} \newline \\
%         \bottomrule
%     \end{tabular}
%     \begin{tabular}{p{0.3\textwidth}p{0.3\textwidth}p{0.3\textwidth}}
%         \toprule
%             \textbf{kShadow: \textit{64.0}} &
%             \textbf{kShadow: \textit{128.0}}   &
%             \textbf{kShadow: \textit{256.0}}   \\
%         \cmidrule(r){1-1}\cmidrule(lr){2-2}\cmidrule(l){3-3}
%             \includegraphics[width=0.3\textwidth]{img/sphere_tracing_shadows_64.pdf} \newline &
%             \includegraphics[width=0.3\textwidth]{img/sphere_tracing_shadows_128.pdf} \newline &
%             \includegraphics[width=0.3\textwidth]{img/sphere_tracing_shadows_256.pdf} \newline \\
%         \bottomrule
%     \end{tabular}
% \end{table}

Das eigentliche Rendering, also die Darstellung der Szene, geschieht in
der Funktion \textit{render}.

\begin{minipage}{\linewidth}
\begin{lstlisting}[language=GLSL,caption={Funktion zur Darstellung der
        Szene in
        GLSL. Die Szene bzw.\ der Farbwert wird nur dann zurückgegeben
        bzw.\ berechnet, wenn eine minimale Distanz nicht unterschritten
        wird.},label={alg:glsl_render},captionpos=b,emph={render}]
// Performs rendering of a scene beginning at given origin in given ray
// direction. This invokes calculating the normal vector, the material
// as well as the lighting.
vec3 render(in vec3 rayOrigin, in vec3 rayDirection)
{
    vec3  color           = vec3(0.05, 0.08, 0.10);
    vec3  res             = castRay(rayOrigin, rayDirection, 100.0, 0.00001, 100);
    float currentDistance = res.x;
    float renderedScene   = res.z;
    float minimalDistance = -0.5;

    // Perform further calculation only when the distance is not below
    // the minimal distance (epsilon-factor).
    if (currentDistance > minimalDistance) {
        vec3 position = rayOrigin + currentDistance * rayDirection;
        vec3 normal   = calcNormal(position, 0.000001);

        vec3 lightColor    = vec3(0.7, 0.2, 0.3);
        vec3 lightPosition = vec3(-0.6, 0.7, -0.5);
        vec3 light         = calcLighting(position, normal, rayDirection, material, lightPosition, lightColor);

        color = light;
    }
    color      = clamp(color, 0.0, 1.0);

    return color;
}
\end{lstlisting}
\end{minipage}

Die Hauptfunktion \textit{main} des Fragment-Shaders ruft die Funktion
zur Darstellung der Szene auf. Alle Methoden (wie z.B.~\textit{getRay}
oder~\textit{squareFrame}) finden sich im Programmcode
des~\hyperref[chap:prototype]{Prototypen}, welcher dieser Projektarbeit
beiliegt.

\begin{minipage}{\linewidth}
\begin{lstlisting}[language=GLSL,caption={Einstiegspunkt des
        Fragment-Shaders.},
    label={alg:glsl_main},captionpos=b,emph={main}]
// Main method of the shader.
void main()
{
    vec2 resolution      = globalResolution;
    float time           = globalTime * 1.1;
    float cameraAngle    = 1.0;
    float cameraHeight   = 2.0;
    float cameraPane     = 4.0;
    float cameraDistance = 4.5;
    vec3 rayOrigin      = vec3(cameraPane * sin(cameraAngle), cameraHeight, cameraDistance * cos(cameraAngle * time));
    vec3 rayTarget      = vec3(0.0, 0.0, 0.0);
    vec2 screenPosition = squareFrame(resolution);
    vec3 rayDirection   = getRay(rayOrigin, rayTarget, screenPosition, 2.0);

    vec3 color          = render(rayOrigin, rayDirection);
    color               = calcPostFx(color, screenPosition);

    gl_FragColor.rgb    = color;
    gl_FragColor.a      = 1.0;
}
\end{lstlisting}
\end{minipage}
