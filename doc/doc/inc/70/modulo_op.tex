% -*- coding: UTF-8 -*-
% vim: autoindent expandtab tabstop=4 sw=4 sts=4 filetype=tex
% vim: spelllang=de spell
% chktex-file 27 - disable warning about missing include files

\section{Repetitions-Operation}
\label{sec:rendering:modulo}

Wie unter~\cite[S. 37ff]{osterwalder_sven_volume_2016} aufgeführt, existieren
für implizite Oberflächen diverse Operationen, wie Distanz- und
Domänen-Operationen sowie -Deformationen.

Möchte man nun aber dasselbe Objekt mehrmals darstellen, so repetiert man
dieses anhand einer oder mehrerer Achsen. Dies geschieht durch Anpassung der
Domäne, also der Anpassung des Raumes, in dem sich eine implizite Oberfläche
befindet. Es handelt sich dabei also um eine Domänen-Operation.

Die Anpassung der Domäne kann mittels der Modulo-Operation (\textit{mod}, \%)
vorgenommen werden. Die Modulo-Operation gibt den (vorzeichenbehafteten) Rest
einer Division zurück~\cite{maignan_integer_2008}.

\begin{gather}
    d(\bm{x}, \bm{c}) = \bm{x} \mod \bm{c} - {\bm{c} \over 2}\\
    \text{repeat}(\bm{x}, \bm{c}) = f(d(\bm{x}, \bm{c}))
\end{gather}

Dabei ist $\bm{x}$ der Punkt einer impliziten Oberfläche $f$ und $\bm{c}$ der
gewünschte Abstand zwischen den Objekten.

Das Objekt wird erst mit der Hälfte des Abstandes $c$, also $c \over 2$, anhand
der Koordinatenachsen verschoben, um es im Rahmen des gewünschten Abstandes
$c$ zu zentrieren.  Danach wird die Repetition der Domäne angewendet. Danach
wird das Objekt zum Ursprung seines Koordinatensystemes zurück verschoben.
