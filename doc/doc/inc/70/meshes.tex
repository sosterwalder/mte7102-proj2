% -*- coding: UTF-8 -*-
% vim: autoindent expandtab tabstop=4 sw=4 sts=4 filetype=tex
% vim: spelllang=de spell
% chktex-file 27 - disable warning about missing include files

\section{Meshes}
\label{sec:rendering:meshes}

Während der Präsentation der vorhergehenden Projektarbeit, MTE7101, kam die
Frage auf, ob es auch möglich ist ``konventionelle'' 3D-Modelle (Meshes) mittels
Sphere-Tracing darzustellen. Die Modellierung bei der Anwendung von Sphere-Tracing findet 
üblicherweise mittels Distanzfunktionen statt und ist daher rein Mathematisch.
Sphere-Tracing nutzt Distanzfunktionen und Distanzfelder zur Darstellung von
impliziten Oberflächen~\cite[S. 31]{osterwalder_sven_volume_2016}.

Betrachtet man nun aber Meshes genauer, so handelt es sich prinzipiell auch um
Tiefeninformationen --- wie bei einem Distanzfeld. Somit müsste es möglich sein
ein Distanzfeld aus einem Mesh zu erzeugen und in einer dreidimensionalen
Textur abzuspeichern. Dazu könnte ein dreidimensionales Raster erzeugt und das
Mesh darin abgelegt werden. Danach müsste man die Distanzwerte von jedem Punkt
des Rasters zur Oberfläche des Meshes in der dreidimensionalen Textur ablegen.
Das Distanzfeld wäre dann zwar initial nicht $\in \R^{3}$, wie für das
Sphere-Tracing benötigt, das Speichern in eine dreidimensionale Textur würde
die Werte jedoch dann wieder auf [0, 1] beschränken.

Dem Autor dieser Arbeit ist bisher kein solches Verfahren bekannt. Ob die oben
beschriebene Methode wirklich funktioniert und wie gut die Resultate wären
müsste ausprobiert werden.

