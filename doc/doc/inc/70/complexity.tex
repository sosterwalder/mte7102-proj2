% -*- coding: UTF-8 -*-
% vim: autoindent expandtab tabstop=4 sw=4 sts=4 filetype=tex
% vim: spelllang=de spell
% chktex-file 27 - disable warning about missing include files

\subsection{Komplexität}
\label{subsec:rendering:modulo:complexity}

Es stellt sich nun die Frage, wie hoch die zusätzliche Komplexität bei
Anwendung der Modulo-Operation ist. Der Autor geht davon aus, dass der Aufwand
zur Darstellung von sich wiederholenden Objekten konstant ist. Die Methode zum
Abtasten der Distanz, \textit{castRay}, ändert sich in diesem Sinne konstant mit
der Anzahl Objekte, die dargestellt werden sollen. Je mehr Objekte, desto höher
der Aufwand. Nimmt man nun an, dass ein Objekt, welches wiederum eine
Komposition von mehreren Objekten sein kann, wiederholt wird, so wird dennoch
nach Erreichen der maximalen Anzahl Schritte oder der maximalen Distanz
abgebrochen, was konstant ist. Vereinfacht gesagt wird schliesslich pro Iteration
nur zurückgegeben, ob auf ein Objekt getroffen wurde oder nicht. Die Berechnung
der Beleuchtung respektiv der Farbe einer Oberfläche erhöht die Komplexität
natürlich nochmals (diese ist aufgrund der Verschachtelung
$\mathcal{O}(n^{2})$). Die Thematik soll an dieser Stelle jedoch nicht weiter
vertieft werden, da dies den Rahmen dieser Projektarbeit deutlich sprengen
würde.
