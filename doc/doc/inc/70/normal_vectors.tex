% -*- coding: UTF-8 -*-
% vim: autoindent expandtab tabstop=4 sw=4 sts=4 filetype=tex
% vim: spelllang=de spell
% chktex-file 27 - disable warning about missing include files

\section{Berechnung von Normalen-Vektoren}
\label{sec:rendering:normals}

Folgender Abschnitt basiert auf~\cite[S. 42 bis
43]{osterwalder_sven_volume_2016}.

Die Oberflächen-Normale kann gemäss~\citeauthor{hart_ray_1989} mittels dem
Gradienten des Distanzfeldes für einen bestimmten Punkt auf einer impliziten
Oberfläche berechnet werden~\parencite[S. 292 bis 293]{hart_ray_1989}.

Verwendet man dies nun zusammen mit den entsprechenden Distanzfunktionen, so
ergeben sich folgende Gleichungen:

\begin{gather}
    \bm{n}_{x} = f(x + \varepsilon, y, z) - f(x - \varepsilon, y, z) \\
    \bm{n}_{y} = f(x, y + \varepsilon,  z) - f(x, y - \varepsilon,  z) \\
    \bm{n}_{z} = f(x, y, z + \varepsilon) - f(x, y, z - \varepsilon) \\
\end{gather}

Dabei ist $\bm{n} = \begin{bmatrix} x_{n} \\ y_{n} \\ z_{n} \end{bmatrix}$ die
Normale der Oberfläche in Form eines Vektors mit drei Komponenten und $f$ eine
Distanzfunktion~\parencites[S. 292 bis 293]{hart_ray_1989}[S.
13]{hart_ray_1993}.

Es werden also insgesamt sechs Punkte abgetastet. $\varepsilon$ ist dabei ein
Wert um alle benachbarten Punkte eines gegebenen Punktes der impliziten
Oberfläche entlang der Koordinatenachsen zu erhalten. Da die Berechnung der
Normalen somit direkt von $\varepsilon$ abhängt, wird für $\varepsilon$
üblicherweise ein möglichst kleiner Wert verwendet.  \citeauthor{hart_ray_1989}
geben $\varepsilon$ als die minimale Inkrementation eines (Licht-) Strahles an.
Die Normale der Oberfläche sollte schliesslich noch normalisiert werden.

``Liefert die oben genannte Gradiente, bestehend aus 6 Punkten, eine zu
geringe Genauigkeit, so kann diese gemäss \citeauthor{hart_ray_1989}
erweitert werden~\parencite[S. 293]{hart_ray_1989}.\\
Die Erweiterung erfolgt durch Hinzunahme von Punkten, welche eine
gemeinsame Kante haben. Dies erzeugt eine Gradiente bestehend aus 18
Punkten. Werden noch die Punkte hinzugenommen, welche gemeinsame
Eckpunkte haben, so ergibt sich eine Gradiente bestehend aus 26
Punkten~\parencite[S. 293]{hart_ray_1989}.''~\cite[S.
43]{osterwalder_sven_volume_2016}

