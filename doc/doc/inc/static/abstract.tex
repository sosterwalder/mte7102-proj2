% -*- coding: UTF-8 -*-
% vim: autoindent expandtab tabstop=4 sw=4 sts=4 filetype=tex
% chktex-file 27

\chapter*{Abstract}
\label{chap:abstract}

Diese Projektarbeit stellt eine Software-Architektur für ein System zur
einfachen Erstellung und Handhabung visueller Szenen in Echtzeit vor.

Das System erlaubt die Erstellung und Handhabung von Szenen mittels einer
grafischen Benutzeroberfläche (GUI). Ein Graph erlaubt eine einfache und
intuitive Komposition von visuellen Szenen. Ein Sequenzer erlaubt die Animation
von Elementen, wie z.B. Modellen oder Bitmaps.

Als Renderingverfahren kommt Sphere Tracing zum Einsatz, ein hoch-optimiertes
Ray-Tracing-Verfahren, welches die Darstellung von Szenen in Echtzeit per GPU
erlaubt.

Die Machbarkeit des Konzeptes wird durch einen Prototypen aufgezeigt, welcher
die Komposition sowie Darstellung von einfachen Szenen in Echtzeit erlaubt.
