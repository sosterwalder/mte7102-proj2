% -*- coding: UTF-8 -*-
% vim: autoindent expandtab tabstop=4 sw=4 sts=4 filetype=tex
% vim: spelllang=de spell
% chktex-file 27 - disable warning about missing include files

\chapter{Aufgabenstellung}
\label{chap:scope}

\section{Motivation}
\label{sec:motivation}

Das in der vorhergehenden
Projektarbeit~\cite{osterwalder_sven_volume_2016} vorgestellte
Renderingverfahren, Sphere Tracing, optimiert das \todo{write ray tracing everywhere
    the same way} Raytracing-Verfahren so weit, dass eine Darstellung von
Szenen mit hoher Qualität in Echtzeit möglich ist.

Das Verfahren nutzt implizite Funktionen um Objekte (und somit auch
Szenen) zu definieren. Selbiges gilt auch für Operationen. Dies mag auf
den ersten Blick praktisch erscheinen, da die Definitionen der meisten Objekte
und Operationen nur wenige Zeilen lang sind. Dies hat jedoch zur Folge, dass
einerseits Programme selbst bei kleinsten Änderungen neu kompiliert werden
müssen und, dass andererseits bei komplexen Szenen schnell die Übersicht
verloren geht. Möchte man Szenen bzw.\ Objekte animieren, so erhöht dies die
Komplexität erneut.

Diese Projektarbeit stellt daher eine Software-Architektur für ein System zur
einfachen Erstellung und Handhabung visueller Szenen in Echtzeit vor.

Die Machbarkeit des \todo{check if Konzept is the right word} Konzeptes wird
durch einen Prototypen aufgezeigt, welcher die Komposition sowie Darstellung
von einfachen Szenen in Echtzeit erlaubt.

\subsection{Demoszene}
\label{subsec:demoscene}

In den 1980er-Jahren entwickelte sich ``unter Anhängern der
Computerszene \dots{} während der Blütezeit der
8-Bit-Systeme''~\parencite{wikipedia_foundation_demoszene_2015} eine Bewegung
namens Demoszene. ``Ihre Mitglieder, die häufig Demoszener oder einfach
Szener genannt werden, erzeugen mit Computerprogrammen auf Rechnern so
genannte Demos – Digitale Kunst, meist in Form von musikalisch
unterlegten
Echtzeit-Animationen.''~\parencite{wikipedia_foundation_demoszene_2015}

Es handelt sich bei der Demoszene um ein sehr aktives und kreatives
Umfeld, in welchem die Technologie ständig an die Grenzen ihrer
Möglichkeiten gebracht wird. In diesem Umfeld entstehen regelmässig neue
Ideen zur Erzeugung von noch realistischer wirkenden Bildern. Dabei
findet eine wechselseitige Beeinflussung zwischen dem akademischen
Umfeld und der Demoszene statt.

So wurde auch das in~\cite{osterwalder_sven_volume_2016} vorgestellte
\textit{Sphere Tracing} Verfahren
relativ früh von in der Demoszene aktiven Personen aufgegriffen und
behandelt.

\section{Ziele und Abgrenzung}
\label{sec:objectives}

Diese Projektarbeit besteht aus zwei Teilen. Der Beschreibung der \todo{add
    ref.\ to sw.\ arch.\ chapter} Software-Architektur sowie der Umsetzung
eines~\hyperref[chap:prototype]{Prototypen}. \todo{Explain, that prototype was
    done before}Dieser bildet einen kleinen Teil
der Software-Architektur ab.

Bei dem entwickelten~\hyperref[chap:prototype]{Prototypen} handelt es sich um
eine Machbarkeitsstudie. Diese zeigt, dass die Komposition sowie Darstellung
von einfachen Szenen in Echtzeit mit dem angedachten System möglich ist.

Diese Projektarbeit dient als Vorarbeit und Grundlage für das MSE-Modul
\textit{MTE7103} --- ``Master Thesis'' (Folgemodul und Abschlussarbeit).

\subsection{Vorgängige Aktivitäten}
\label{subsec:preliminaries}

Die vorhergehende Projektarbeit~\cite{osterwalder_sven_volume_2016} bildet die
Grundlage für diese Projektarbeit.

\subsection{Neue Lerninhalte}
\label{subsec:new_learning_contents}

Zusätzlich zu den formalen Lerninhalten hatte die Arbeit für den Autor
die folgenden \todo{Provide new learning contents} neuen Lerninhalte:
\begin{itemize}
    \item{TODO}
\end{itemize}
