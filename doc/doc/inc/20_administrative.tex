%* -*- coding: UTF-8 -*-
% vim: autoindent expandtab tabstop=4 sw=4 sts=4 filetype=tex
% vim: spelllang=de spell
% chktex-file 27 - disable warning about missing include files

\chapter{Administratives}
\label{chap:20_administrative}

Einige administrative Aspekte der Projektarbeit werden angesprochen,
obwohl sie für das Verständnis der Resultate nicht notwendig sind.

Im gesamten Dokument wird nur die männliche Form verwendet, womit aber
beide Geschlechter gemeint sind.

\section{Beteiligte Personen}
\label{sec:involved_persons}

\begin{tabbing} %tabulator
Bezeichnung: \= Name name name name name name\= Zuständigkeit \kill \\
    Autor           \> Sven Osterwalder\protect\footnotemark[1]{}    \> \\
    Betreuer        \> Prof.\ Claude Fuhrer\protect\footnotemark[2]{}  \> \textit{Begleitet den Studenten bei der Projektarbeit}\\
\end{tabbing}
\footnotetext[1]{sven.osterwalder@students.bfh.ch}
\footnotetext[2]{claude.fuhrer@bfh.ch}

\section{Aufbau des Dokumentes}
\label{sec:document_structure}

Die vorliegende Arbeit ist aufgebaut wie folgt:
\begin{itemize}
    \item Einleitung zur Projektarbeit
    \item Beschreibung der Aufgabenstellung
    \item Vorgehen des Autors im Hinblick auf die gestellten Aufgaben
    \item Lösung der gestellten Aufgaben
    \item Verwendete Technologien
\end{itemize}

\section{Ergebnisse (Deliverables)}
\label{sec:deliverables}

Nachfolgend sind die abzugebenden Objekte aufgeführt:
\begin{itemize}
    \item \textbf{Abschlussdokument} \\
        Das Abschlussdokument beinhaltet die Ausarbeitung einer
        Software-Architektur eines Systems zur einfachen Erstellung und
        Handhabung visueller Szenen in Echtzeit.
    \item \textbf{Protoyp} \\
        Der~\hyperref[chap:prototype]{Prototyp} entstand im Rahmen der
        Bearbeitung der Thematik. Er zeigt die Machbarkeit (von Teilen) der
        angedachten Software-Architektur auf.
\end{itemize}
