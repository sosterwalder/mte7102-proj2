% -*- coding: UTF-8 -*-
% vim: autoindent expandtab tabstop=4 sw=4 sts=4 filetype=tex
% vim: spelllang=en spell

\documentclass[conference]{IEEEtran}
\begin{document}

% Do not put math or special symbols in the title.
\title{QDE --- A visual animation system}

% author names and affiliations
% use a multiple column layout for up to three different
% affiliations
\author{\IEEEauthorblockN{Sven Osterwalder}
\IEEEauthorblockA{Bern University of Applied Sciences\\ BFH-TI, CH2502-Biel, Switzerland.\\
Email: sven.osterwalder@students.bfh.ch}}

% conference papers do not typically use \thanks and this command
% is locked out in conference mode. If really needed, such as for
% the acknowledgment of grants, issue a \IEEEoverridecommandlockouts
% after \documentclass

% for over three affiliations, or if they all won't fit within the width
% of the page, use this alternative format:

% make the title area
\maketitle

% As a general rule, do not put math, special symbols or citations
% in the abstract
\begin{abstract}
\noindent{}%
A software architecture for a visual animation system is presented. The system
allows the creation of visually appealing scenes using a graphical user
interface. A node-based graph structure and a sequencer allow compositing and
animating elements such as models or bitmaps. For rendering a highly optimized
algorithm based on ray tracing is used, which allows the rendering of ray
traced scenes in real-time on the GPU.\@ The concept of the node-based graph
structure is demonstrated using a prototype application which allows
compositing a simple scene using primitives and render it in real-time.
\end{abstract}

% no keywords

\IEEEpeerreviewmaketitle{}

\end{document}
